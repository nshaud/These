%%%%%%%%%%%%%%%%%%%%%%%%%%%%%%%%%%%%%%%%%%%%%%%%%%%%%
%%                   Acronymes			   %%
%%%%%%%%%%%%%%%%%%%%%%%%%%%%%%%%%%%%%%%%%%%%%%%%%%%%%

\newacronym{ISPRS}{ISPRS}{International Society for Photogrammetry and Remote Sensing}

\newacronym{DFC}{DFC}{Data Fusion Contest}

\newacronym{IEEE}{IEEE}{Institute of Electrical and Electronics Engineers}

\newacronym{GRSS}{GRSS}{Geoscience \& Remote Sensing Society}

\newacronym{DBN}{DBN}{\emph{Deep Belief Networks}}

\newglossaryentry{CNN}{
	type=\acronymtype,
	name={CNN},
	first={réseau de neurones convolutif, ou \emph{Convolutional Neural Network} (CNN)},
	description={\emph{Convolutional Neural Network}, réseau de neurones dont certaines couches effectuent des convolutions à poids partagés}
	}

\newglossaryentry{FCN}{
	type=\acronymtype,
	name={FCN},
	first={réseau de neurones entièrement convolutif, ou \emph{Fully Convolutional Network} (FCN)},
	description={\emph{Fully Convolutional Network}, réseau de neurones entièrement convolutif conservant la structure spatiale grâce à l'absence de couches entièrement connectées}
	}

\newglossaryentry{SVM}{
  type=\acronymtype,
  name={SVM},
  first={Séparateur à Vaste Marge, ou \emph{Support Vector Machine} (SVM), aussi appelées machines à vecteur de support,},
  description={Méthode de classification}
  }

\newglossaryentry{MNS}{
  type=\acronymtype,
  name={MNS},
  first={Modèle Numérique de Surface (MNS)},
  description={Description de la hauteur des objets surélevés par rapport au terrain sous-jacent}
}

\newglossaryentry{MNE}{
	type=\acronymtype,
	name={MNE},
	first={Modèle Numérique d'Élévation (MNE)},
	description={Modélisation de la topographie d'une surface terrestre prenant en compte les objets surélevés}
}

\newglossaryentry{MNT}{
  type=\acronymtype,
  name={MNT},
  first={Modèle Numérique de Terrain (MNT)},
  description={Modélisation de la topographie d'une surface terrestre, ne prenant pas en compte les objets surélevés}
}

\newglossaryentry{Lidar}{
  type=\acronymtype,
  name={Lidar},
  first={\emph{light detection and ranging} (Lidar)},
  description={Technique de mesure de distance utilisant le temps parcouru par un faisceau lumineux entre son émission et la réception de son écho}
}

\newglossaryentry{CUDA}{
  type=\acronymtype,
  name={CUDA},
  first={\emph{Compute Unified Device Architecture} (CUDA)},
  description={\emph{Compute Unified Device Architecture}, technologie permettant de programmer sur \gls{GPU} depuis le langage C}
}

\newglossaryentry{GPU}{
	type=\acronymtype,
	name={GPU},
	first={\emph{Graphics Processing Unit} (GPU)},
	description={\emph{Graphics Processing Unit}, processeur de calcul hautement parallèle originellement conçu pour le rendu graphique, puis réutilisé pour le calcul scientifique matriciel, notamment dans le cas des réseaux convolutifs profonds}
}

\newacronym{ILSVRC}{ILSVRC}{\emph{ImageNet Large-Scale Visual Recognition Challenge}}

\newacronym{ReLU}{ReLU}{\emph{Rectified Linear Unit}}

\newacronym{PReLU}{PReLU}{\emph{Parametrized Rectified Linear Unit}}

\newacronym{ELU}{ELU}{\emph{Exponential Linear Unit}}

\newglossaryentry{TIFF}{
	type=\acronymtype,
	name={TIFF},
	first={TIFF},
	description={\emph{Tagged Image File Format}, format de fichier image supportant l'ajout d'informations de géoréférencement grâce au standard GeoTIFF}
}
% SENTINEL
% SPOT
% LANDSAT
% IGN


%%%%%% Segmentations

\newglossaryentry{SLIC}{
  type=\acronymtype,
  name={SLIC},
  first={\emph{Simple Linear Iterative Clustering} (SLIC)},
  description={Algorithme de segmentation superpixels proposé dans~\cite{achanta_slic_2010}}
}

\newglossaryentry{SEEDS}{
  type=\acronymtype,
  name={SEEDS},
  first={\emph{Superpixels Extracted via Energy-Driven Sampling} (SEEDS)},
  description={Algorithme de segmentation superpixels proposé dans~\cite{bergh_seeds:_2012}}
}

\newglossaryentry{ERS}{
	type=\acronymtype,
	name={ERS},
	first={\emph{} (ERS)},
	description={Algorithme de segmentation par marche aléatoire proposé dans~\cite{}}
}

\newglossaryentry{LSC}{
  type=\acronymtype,
  name={LSC},
  first={\emph{Linear Spectral Clustering} (LSC)},
  description={Algorithme de segmentation superpixels proposé dans~\cite{li_superpixel_2015}}
}

\newglossaryentry{MRS}{
  type=\acronymtype,
  name={MRS},
  first={\emph{Multi-Resolution Segmentation} (MRS)},
  description={Algorithme de segmentation multi-échelles proposé par~\cite{baatz_multiresolution_2010} et implémenté dans le logiciel eCognition}
}

\newglossaryentry{FH}{
  type=\acronymtype,
  name={FH},
  first={Felzenszwalb-Huttenlocher (FH)},
  description={Algorithme de segmentation d'image proposé dans~\cite{felzenszwalb_efficient_2004}}
}
\newglossaryentry{IsU}{
  type=\acronymtype,
  name={IsU},
  first={intersection sur union (IsU)},
  description={Mesure de performance d'un classifieur utilisée notamment dans le cadre de la segmentation sémantique. Elle correspond au rapport du nombre d'échantillons étant positifs à la fois pour la classifieur et en réalité et du nombre d'échantillons étant positifs dans le classifieur ou dans la réalité}
}

\newglossaryentry{SCALP}{
  type=\acronymtype,
  name={SCALP},
  first={\emph{Superpixels with Contour Adherence using Linear Path} (SCALP)},
  description={Algorithme de segmentation de type superpixels introduit dans~\cite{giraud_robust_2018}}
}


\newglossaryentry{HSEG}{
  type=\acronymtype,
  name={HSeg},
  first={\emph{Hierarchial Segmentation} (HSeg)},
  description={Algorithme de segmentation hiérarchique proposé dans~\cite{tilton_best_2012}}
}

%%%%%% COULEURS %%%%%%

\newglossaryentry{IRRV}{
  type=\acronymtype,
  name={IRRV},
  first={infra-rouge-rouge-vert (IRRV)},
  description={Image trois canaux dans le proche infrarouge, le rouge et le vert}
}

\newglossaryentry{RVB}{
  type=\acronymtype,
  name={RVB},
  first={rouge-vert-bleu (RVB)},
  description={Espace de représentation des images naturelles sous forme de trois canaux rouge, vert et bleu}
}

\newglossaryentry{IRRVB}{
  type=\acronymtype,
  name={IRRVB},
  first={infrarouge-rouge-vert-bleu (IRRVB)},
  description={Image multispectrale dans le proche infrarouge, le rouge, le vert et le bleu}
}

\newglossaryentry{LAB}{
	type=\acronymtype,
	name={CIELAB},
	first={L*a*b* CIE 1976 (CIELAB)},
	description={L'espace L*a*b* CIE 1976, ou CIELAB, est un espace de représentation de couleurs utilisant la clarté $L*$, dérivée de la luminance, et $a*$ et $b*$ comme paramètres exprimant l'écart colorimétrique par rapport à une surface grise. L'intérêt de cette représentation est qu'une distance euclidienne dans l'espace CIELAB est proche de la distance colorimétrique perçue par l'\oe{}il humain}
}

\newglossaryentry{HR}{
  type=\acronymtype,
  name={HR},
  first={haute résolution (HR)},
  description={Désigne une image de télédétection d'une résolution inférieure au sol à $1m$}
}

\newglossaryentry{THR}{
  type=\acronymtype,
  name={THR},
  first={très haute résolution (THR)},
  description={Désigne une image de télédétection d'une résolution inférieure au sol à $50cm$}
}
\newglossaryentry{EHR}{
  type=\acronymtype,
  name={EHR},
  first={extrêmement haute résolution (EHR)},
  description={Désigne une image de télédétection d'une résolution inférieure au sol à $10cm$}
}

%%%%%
\newglossaryentry{HOG}{
  type=\acronymtype,
  name={HOG},
  first={histogrammes de gradient orientés},
  description={Approximation discrète de la distribution locale du gradient dans une image selon une direction pré-définie. Il s'agit d'une caractéristique image introduite dans~\cite{dalal_histograms_2005}}
}

\newglossaryentry{SIFT}{
	type=\acronymtype,
	name={SIFT},
	first={SIFT \emph{(scale-invariant feature transform)}},
	description={Les descripteurs SIFT \emph{(scale-invariant feature transform)} sont des caractéristiques images calculées sur des points d'intérêt cherchant une invariance à l'échelle, l'angle de vue et à la luminosité}
}

%%%%%%%%%%%%%%%%%%%%%%%%%%%%%%%%%%%%%%%%%%%%%%%%%%%%%
%%	Définitions (glossaires standard)	   %%
%%%%%%%%%%%%%%%%%%%%%%%%%%%%%%%%%%%%%%%%%%%%%%%%%%%%%

\newglossaryentry{teledetection}{
	name={télédétection},
	description={Ensemble des techniques utilisées pour caractériser la surface de la Terre à partir de mesures aéroportées ou satellitaires},
	}

\newglossaryentry{apprentissageautomatique}{
	name={apprentissage automatique},
	description={Ensemble des techniques mises en \oe{}uvre pour la génération automatique d'algorithmes de résolution de problèmes},
	}

\newglossaryentry{multispectral}{
        name={multispectral},
        description={Imagerie utilisant des récepteurs sur plusieurs longueurs d'onde},
}

\newglossaryentry{hyperspectral}{
        name={hyperspectral},
        description={Imagerie utilisant des récepteurs sur plusieurs dizaines de longueurs d'onde, y compris hors du domaine visible}
}

%%%%%%%%%%%%%%%%%%%%%%%%%%%%%%%%%%%%%%%%%%%%%%%%%%%%%
%%					Symboles						%
%%%%%%%%%%%%%%%%%%%%%%%%%%%%%%%%%%%%%%%%%%%%%%%%%%%%%
\newglossaryentry{alpha}{
  type=notation,
  name={\ensuremath{\alpha}},
  description={Alpha est une variable},
  sort={alpha}%
}

\newglossaryentry{gamma}{
  type=notation,
  name={\ensuremath{\gamma}},
  description={Gamma est un paramètre},
  sort={gamma}
}
