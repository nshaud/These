%%%%%%%%%%%%%%%%%%%%%%%%%%%%%%%%%%%%%%%%%%%%%%%%%%%%%
%%                   Acronymes			   %%
%%%%%%%%%%%%%%%%%%%%%%%%%%%%%%%%%%%%%%%%%%%%%%%%%%%%%

\newacronym[symbol={ISPRS}]{ISPRS}{ISPRS}{\emph{International Society for Photogrammetry and Remote Sensing}}

\newacronym[symbol={DFC}]{DFC}{DFC}{\emph{Data Fusion Contest}}

\newacronym{IEEE}{IEEE}{\emph{Institute of Electrical and Electronics Engineers}}

\newacronym{GRSS}{GRSS}{\emph{Geoscience \& Remote Sensing Society}}

\newacronym{DBN}{DBN}{\emph{Deep Belief Networks}}

\newacronym{RBM}{RBM}{\emph{Restricted Boltzmann Machines}}

\newacronym{LRN}{LRN}{\emph{Local Response Normalization}}

\newacronym{AVIRIS}{AVIRIS}{\emph{Airborne Visible/Infrared Imaging Spectrometer}}

\newacronym[symbol={NN}]{NN}{NN}{\emph{Neural Network}}

\newacronym[symbol={RNN}]{RNN}{RNN}{\emph{Recurrent Neural Network}}

\newglossaryentry{CNN}{
	type=\acronymtype,
	name={CNN},
	symbol={CNN},
	plural={CNN},
	first={réseau de neurones convolutif, ou \emph{Convolutional Neural Network} (CNN)},
	firstplural={réseaux de neurones convolutifs, ou \emph{Convolutional Neural Networks} (CNN)},
	description={\emph{Convolutional Neural Network}, réseau de neurones dont certaines couches effectuent des convolutions à poids partagés}
	}

\newglossaryentry{GAN}{
	type=\acronymtype,
	name={GAN},
	plural={GAN},
	firstplural={réseaux génératifs adversaires, ou \emph{Generative Adversarial Networks} (GAN)},
	first={réseau génératif adversaire, ou \emph{Generative Adversarial Network} (GAN)},
	description={modèle génératif utilisant deux réseaux de neurones entraînés en concurrence}
	}

\newglossaryentry{FCN}{
	type=\acronymtype,
	name={FCN},
	first={réseau de neurones entièrement convolutif, ou \emph{Fully Convolutional Network} (FCN)},
	firstplural={réseaux de neurones entièrement convolutifs, ou \emph{Fully Convolutional Networks} (FCN)},
	plural={FCN},
	symbol={FCN},
	description={\emph{Fully Convolutional Network}, réseau de neurones entièrement convolutif conservant la structure spatiale grâce à l'absence de couches entièrement connectées}
	}

\newglossaryentry{SVM}{
  type=\acronymtype,
  name={SVM},
	plural={SVM},
	symbol={SVM},
  first={machine à vecteurs de support, ou \emph{Support Vector Machine} (SVM),},
	firstplural={machines à vecteurs de support, ou \emph{Support Vector Machines} (SVM),},
  description={\emph{Support Vector Machine}, en français machine à vecteurs de support, parfois sous la dénomination Séparateur à Vaste Marge. Outil de classification ou de régression.}
  }

\newglossaryentry{MNH}{
  type=\acronymtype,
  name={MNH},
  first={Modèle Numérique de Hauteur (MNH)},
  description={Modèle Numérique de Hauteur. Mesure de la hauteur des points surélevés par normalisée par rapport au terrain sous-jacent}
}

\newglossaryentry{MNE}{
	type=\acronymtype,
	name={MNE},
	first={Modèle Numérique d'Élévation (MNE)},
	description={Modèle Numérique d'Élévation. Description de la topographie d'une surface terrestre prenant en compte les objets surélevés. Parfois également nommé modèle numérique de surface (MNS) dans la littérature}
}

\newglossaryentry{MNT}{
  type=\acronymtype,
  name={MNT},
  first={Modèle Numérique de Terrain (MNT)},
  description={Modèle Numérique de Terrain. Description de la topographie d'une surface terrestre, ne prenant pas en compte les objets surélevés}
}

\newglossaryentry{Lidar}{
  type=\acronymtype,
  name={Lidar},
  first={\emph{light detection and ranging} (Lidar)},
  description={\emph{Light Detection And Ranging}, technique de mesure de distance utilisant le temps parcouru par un faisceau lumineux entre son émission et la réception de son écho}
}

\newglossaryentry{CUDA}{
  type=\acronymtype,
  name={CUDA},
  first={\emph{Compute Unified Device Architecture} (CUDA)},
  description={\emph{Compute Unified Device Architecture}, technologie permettant de programmer sur \gls{GPU} depuis le langage C}
}

\newglossaryentry{CPU}{
	type=\acronymtype,
	name={CPU},
	symbol={CPU},
	first={\emph{Central Processing Unit} (CPU)},
	description={\emph{Central Processing Unit}, processeur de calcul générique d'un ordinateur}
}

\newglossaryentry{GPU}{
	type=\acronymtype,
	name={GPU},
	symbol={GPU},
	first={\emph{Graphics Processing Unit} (GPU)},
	description={\emph{Graphics Processing Unit}, processeur de calcul hautement parallèle originellement conçu pour le rendu graphique, puis réutilisé pour le calcul scientifique matriciel, notamment dans le cas des réseaux convolutifs profonds}
}

\newacronym{ILSVRC}{ILSVRC}{\emph{ImageNet Large-Scale Visual Recognition Challenge}}

\newacronym{ReLU}{ReLU}{\emph{Rectified Linear Unit}}

\newacronym{PReLU}{PReLU}{\emph{Parametrized Rectified Linear Unit}}

\newacronym{ELU}{ELU}{\emph{Exponential Linear Unit}}

\newglossaryentry{TIFF}{
	type=\acronymtype,
	name={TIFF},
	first={TIFF},
	description={\emph{Tagged Image File Format}, format de fichier image supportant l'ajout d'informations de géoréférencement grâce au standard GeoTIFF}
}

\newglossaryentry{SPOT}{
  type=\acronymtype,
	name={SPOT},
	first={SPOT (Satellite pour l'observation de la Terre)},
	description={Satellite pour l'observation de la Terre, famille de satellites de télédétection français conçus par le \gls{CNES}}
}

\newglossaryentry{Landsat}{
	name={Landsat},
	first={Landsat},
	description={Premier programme spatial d'observation de la Terre, initié par la NASA en 1972.}
}

\newglossaryentry{Sentinel}{
	name={Sentinel},
	first={Sentinel},
	description={Programme spatial d'observation de la Terre européen, démarré en 2014 avec les satellites \gls{SAR} Sentinel-1A/B. La famille Sentinel comporte également les satellites multispectraux Sentinel-2A/B lancés en 2015 et 2017, et les satellites Sentinel-3A/B lancés en 2016 et 2018 pour l'océanographie.}
}

\newglossaryentry{Pleiades}{
  name={Pléiades},
	first={la constellation de satellites Pléiades},
	description={Duo de satellites optiques très haute résolution français}
}

\newglossaryentry{PyTorch}{
	name={PyTorch},
	first={PyTorch},
	description={Bibliothèque logicielle C++/Python de calcul tensoriel sur \gls{CPU} et \gls{GPU}, spécialisée pour l'apprentissage profond~\cite{noauthor_pytorch_2016}.}
}

\newglossaryentry{Caffe}{
	name={Caffe},
	first={Caffe},
	description={Bibliothèque logicielle C++ dotée d'interfaces Python et Matlab pour l'apprentissage profond.}
}

\newglossaryentry{scikit-learn}{
  name={scikit-learn},
	first={scikit-learn},
	description={Bibliothèque logicielle Python d'apprentissage automatique.}
}

\newacronym{SAR}{SAR}{radar à synthèse d'ouverture (en anglais \emph{synthetic aperture radar})}

\newacronym{CNES}{CNES}{Centre national d'études spatiales}

\newacronym{IGN}{IGN}{Institut national de l'information géographique et forestière}

\newacronym{ONERA}{ONERA}{Office national d'études et de recherches aérospatiales}

\newacronym{RGB-D}{RGB-D}{Red-Green-Blue + Depth}

\newacronym{VEDAI}{VEDAI}{\emph{Vehicle Detection in Aerial Imagery}}

\newacronym[shortplural={SIG},longplural=systèmes d'information géographiques]{SIG}{SIG}{système d'information géographique}

\newacronym{OSM}{OSM}{\emph{OpenStreetMap}}

\newacronym{ACP}{ACP}{analyse en composantes principales}
% SENTINEL
% SPOT
% LANDSAT
% IGN


%%%%%% Segmentations

\newglossaryentry{SLIC}{
  type=\acronymtype,
  name={SLIC},
  first={\emph{Simple Linear Iterative Clustering} (SLIC)},
  description={\emph{Simple Linear Iterative Clustering}. Algorithme de segmentation superpixels proposé dans~\cite{achanta_slic_2010}}
}

\newglossaryentry{SEEDS}{
  type=\acronymtype,
  name={SEEDS},
  first={\emph{Superpixels Extracted via Energy-Driven Sampling} (SEEDS)},
  description={\emph{Superpixels Extracted via Energy-Driven Sampling}. Algorithme de segmentation superpixels proposé dans~\cite{van_den_bergh_seeds_2012}}
}

\newglossaryentry{ERS}{
	type=\acronymtype,
	name={ERS},
	first={\emph{} (ERS)},
	description={Algorithme de segmentation par marche aléatoire proposé dans~\cite{}}
}

\newglossaryentry{LSC}{
  type=\acronymtype,
  name={LSC},
  first={\emph{Linear Spectral Clustering} (LSC)},
  description={\emph{Linear Spectral Clustering}. Algorithme de segmentation superpixels proposé dans~\cite{li_superpixel_2015}}
}

\newglossaryentry{MRS}{
  type=\acronymtype,
  name={MRS},
  first={\emph{Multi-Resolution Segmentation} (MRS)},
  description={\emph{Multi-Resolution Segmentation}. Algorithme de segmentation multi-échelles proposé par~\cite{baatz_multiresolution_2010} et implémenté dans le logiciel eCognition}
}

\newglossaryentry{FH}{
  type=\acronymtype,
  name={FH},
  first={Felzenszwalb-Huttenlocher (FH)},
  description={Algorithme de segmentation d'image proposé par Felzenszwalb et Huttenlocher~\cite{felzenszwalb_efficient_2004}}
}
\newglossaryentry{IsU}{
  type=\acronymtype,
  name={IsU},
	symbol={IsU},
  first={intersection sur union (IsU)},
  description={Intersection sur union. Mesure de performance d'un classifieur utilisée notamment dans le cadre de la segmentation sémantique. Elle correspond au rapport du nombre d'échantillons étant positifs à la fois pour la classifieur et en réalité et du nombre d'échantillons étant positifs dans le classifieur ou dans la réalité}
}

\newglossaryentry{SCALP}{
  type=\acronymtype,
  name={SCALP},
  first={\emph{Superpixels with Contour Adherence using Linear Path} (SCALP)},
  description={\emph{Superpixels with Contour Adherence using Linear Path}. Algorithme de segmentation de type superpixels introduit dans~\cite{giraud_robust_2018}}
}


\newglossaryentry{HSEG}{
  type=\acronymtype,
  name={HSeg},
  first={\emph{Hierarchial Segmentation} (HSeg)},
  description={\emph{Hierarchial Segmentation}. Algorithme de segmentation hiérarchique proposé dans~\cite{tilton_best_2012}}
}

%%%%%% COULEURS %%%%%%

\newglossaryentry{IRRV}{
  type=\acronymtype,
  name={IRRV},
	symbol={IRRV},
  first={infra-rouge-rouge-vert (IRRV)},
  description={Image trois canaux dans le proche infrarouge, le rouge et le vert}
}

\newglossaryentry{RVB}{
  type=\acronymtype,
  name={RVB},
	symbol={RVB},
  first={rouge-vert-bleu (RVB)},
  description={Espace de représentation des images naturelles sous forme de trois canaux rouge, vert et bleu}
}

\newglossaryentry{IRRVB}{
  type=\acronymtype,
  name={IRRVB},
  first={infrarouge-rouge-vert-bleu (IRRVB)},
  description={Image multispectrale dans le proche infrarouge, le rouge, le vert et le bleu}
}

\newglossaryentry{LAB}{
	type=\acronymtype,
	name={CIELAB},
	first={L*a*b* CIE 1976 (CIELAB)},
	description={L'espace L*a*b* CIE 1976, ou CIELAB, est un espace de représentation de couleurs utilisant la clarté $L*$, dérivée de la luminance, et $a*$ et $b*$ comme paramètres exprimant l'écart colorimétrique par rapport à une surface grise. L'intérêt de cette représentation est qu'une distance euclidienne dans l'espace CIELAB est proche de la distance colorimétrique perçue par l'\oe{}il humain}
}

\newglossaryentry{HR}{
  type=\acronymtype,
  name={HR},
  first={haute résolution (HR)},
  description={Haute résolution. Désigne une image de télédétection d'une résolution inférieure au sol à $1m$}
}

\newglossaryentry{THR}{
  type=\acronymtype,
  name={THR},
  first={très haute résolution (THR)},
  description={Très haute résolution. Désigne une image de télédétection d'une résolution inférieure au sol à $50cm$}
}
\newglossaryentry{EHR}{
  type=\acronymtype,
  name={EHR},
  first={extrêmement haute résolution (EHR)},
  description={Extrêmement haute résolution. Désigne une image de télédétection d'une résolution inférieure au sol à $10cm$}
}

%%%%%
\newglossaryentry{HOG}{
  type=\acronymtype,
  name={HOG},
  first={histogrammes de gradient orientés (HOG)},
  description={\emph{Histograms of Oriented Gradients} (HOG) ou histogrammes de gradients orientés. Approximation discrète de la distribution locale du gradient dans une image selon une direction pré-définie. Il s'agit d'une caractéristique image introduite dans~\cite{dalal_histograms_2005}}
}

\newglossaryentry{SIFT}{
	type=\acronymtype,
	name={SIFT},
	first={SIFT \emph{(scale-invariant feature transform)}},
	description={Les descripteurs SIFT \emph{(scale-invariant feature transform)} sont des caractéristiques images calculées sur des points d'intérêt cherchant une invariance à l'échelle, l'angle de vue et à la luminosité}
}

\newacronym{JPEG}{JPEG}{\emph{Joint Photographic Experts Group}}

\newacronym[symbol={CRF},shortplural={CRF},longplural={champs de Markov conditionnels, ou \emph{Conditional Random Fields}}]{CRF}{CRF}{champ de Markov conditionnel, ou \emph{Conditional Random Field}}

\newacronym{NDVI}{NDVI}{\emph{Normalized Difference Vegetation Index}}

\newacronym{NDWI}{NDWI}{\emph{Normalized Difference Water Index}}

\newacronym[symbol={CDS},shortplural={CDS},longplural={cartes de distances signées}]{CDS}{CDS}{carte de distance signée}

%%%%%%%%%%%%%%%%%%%%%%%%%%%%%%%%%%%%%%%%%%%%%%%%%%%%%
%%	Définitions (glossaires standard)	   %%
%%%%%%%%%%%%%%%%%%%%%%%%%%%%%%%%%%%%%%%%%%%%%%%%%%%%%

\newglossaryentry{teledetection}{
	name={télédétection},
	description={Ensemble des techniques utilisées pour caractériser la surface de la Terre à partir de mesures aéroportées ou satellitaires},
	}

\newglossaryentry{apprentissageautomatique}{
	name={apprentissage automatique},
	description={Ensemble des techniques mises en \oe{}uvre pour la génération automatique d'algorithmes de résolution de problèmes},
	}

\newglossaryentry{multispectral}{
        name={multispectral},
				first={caméra multispectrale},
				plural={multispectrales},
				symbol={MSI},
        description={Imagerie utilisant des récepteurs sur plusieurs longueurs d'onde},
}

\newglossaryentry{hyperspectral}{
        name={hyperspectral},
				first={imagerie hyperspectrale},
				plural={hyperspectrales},
        description={Imagerie utilisant des récepteurs sur plusieurs dizaines de longueurs d'onde, y compris hors du domaine visible}
}

%%%%%%%%%%%%%%%%%%%%%%%%%%%%%%%%%%%%%%%%%%%%%%%%%%%%%
%%					Symboles						%
%%%%%%%%%%%%%%%%%%%%%%%%%%%%%%%%%%%%%%%%%%%%%%%%%%%%%
\newglossaryentry{alpha}{
  type=notation,
  name={\ensuremath{\alpha}},
  description={Alpha est une variable},
  sort={alpha}%
}

\newglossaryentry{gamma}{
  type=notation,
  name={\ensuremath{\gamma}},
  description={Gamma est un paramètre},
  sort={gamma}
}
