\chapter{Segmentation sémantique multi-modale}

\section{Apprentissage multi-modal}

\subsection{Approches en vision par ordinateur}

Principalement autour du RGB+profondeur

Empilement ne donne rien

Approches de fusion a posterio, par exemple auto-encodeurs audio/vidéo

Approches d'apprentissage conjoint, par exemple FuseNet

\subsection{Transposition à la télédétection}

facile

\section{Fusion de modèles}

\subsection{Mélanges de modèles}

Approches a posteriori : moyenne ou combinaison apprise

Similaire à de l'apprentissage par ensemble

\subsection{Fusion par apprentissage}

Laisser le modèle trouver la combinaison intéressante : ~moyenne avec pondération adaptative

Approche résiduelle : correction d'erreur

\section{Connaissances \textit{a priori}}

\subsection{OpenStreetMap}

Présentation d'OSM

Rasterisation

\subsection{Information \textit{a priori} comme capteur virtuel}

Utilisation comme capteur virtuel

Apport sur des classes connues *et* inconnues !
