\chapter{Introduction}
	\citationChap{}{}
	\minitoc
	\newpage


% Début du chapitre
\section{Problématique}
L'observation est à la base de la méthode scientifique. Tout objet d'intérêt est susceptible d'être examiné sous toutes ses coutures pour en comprendre les propriétés. La Terre ne fait pas exception à cette logique. Ce n'est donc pas une surprise si, lors de l'avènement des premiers programmes spatiaux, les premiers satellites mis en orbite étaient résolument tournés vers notre planète. En effet, l'altitude offre une toute nouvelle perspective, d'une part par sa position verticale toute particulière, d'autre part par son champ de vision inégalable.
L'imagerie aérienne et satellite est donc devenu un outil particulièrement présent dans les sciences modernes. Comprendre la Terre est un enjeu scientifique majeur, et l'observation est le premier pas nécessaire à toute tentative de modélisation. Qu'il s'agisse de météorologie, d'océanographie, d'écologie ou de géographie, les images aéroportées fournissent une information formidablement riche.

L'intensification des efforts pour imager la Terre dans son intégralité à haute fréquence n'a donc rien d'étonnant. Des initiatives telles que Landsat, Sentinel ou SPOT permettent d'imager le globe tout entier presque chaque semaine. Pour autant, exploiter cette masse de données n'est pas chose aisée. Interpréter et comprendre une image satellite nécessite une expertise forte, aussi bien liée au capteur utilisé pour acquérir l'image qu'aux connaissances relevant du champ d'application visé.

Ainsi, malgré leur nombre, les photo-interprètes ne peuvent assumer seuls cette responsabilité. L'automatisation présente alors une alternative intéressante. Conférer aux machines la capacité d'interpréter les images de la Terre permettrait de multiplier les observations, pour en tirer à la fois informations et modèles.

C'est le sujet de cette thèse.

Nous cherchons à concevoir, implémenter et de valider des modèles de réseaux de neurones artificiels profonds pour l'interprétation automatisée d'images aériennes et satellites, issues de capteurs multiples, sur une large variété de scènes et pour différents champs d'application.

\section{Domaine}

\section{Contexte}


\bibliographystyle{francaissc}
\bibliography{Chapitre1/Biblio}
