\chapter{Extension aux capteurs non-conventionnels}

\section{Images multispectrales}

\subsection{Prise en compte du proche infrarouge}

De nombreuses images comportent du proche infrarouge
=> la littérature propose de le traiter en IRRG (image composite) et ça marche bien

Si on fait de l'IRRGB, sur Potsdam ça marche moins bien...

\subsection{Images satellites}

On peut traiter du multispectral : exemple, Sentinel-2 pour classif GlobeCover
Ça marche pas mal

\section{Imagerie laser et modèles de terrain}

\subsection{Modèle de terrain}

Modèle numérique de terrain, modèle numérique de surface, modèle numérique de surface normalisé

Traitement direct du modèle numérique de terrain avec un FCN

\subsection{Construction d'une image composite}

Quelles propriétés on aimerait pour couvrir toutes les classes ?
  - hauteur
  - végétation

Image composite : DSM+NDSM+NDVI

\section{Imagerie hyperspectrale}

\subsection{Méthodes pixelliques}

Bref état de l'art

Classification par NN 1D, CNN 1D

\subsection{Méthodes spatiales-spectrales}

Méthodes 2D : CNN 2.5D, ACP puis CNN 2D

Méthodes 3D : CNN 3D

