%!TEX root = Manuscrit.tex
%%%%%%%%%%%%%%%%%%%%%%%%%%%%%%%%%%%%%%%%%%
%           Page de Garde		 %
%%%%%%%%%%%%%%%%%%%%%%%%%%%%%%%%%%%%%%%%%%

\makeatletter
\def\@ecole{école}
\newcommand{\ecole}[1]{
  \def\@ecole{#1}
}

\def\@titleen{titleen}
\newcommand{\titleen}[1]{
  \def\@titleen{#1}
}

\def\@specialite{Spécialité}
\newcommand{\specialite}[1]{
  \def\@specialite{#1}
}

\def\@directeur{directeur}
\newcommand{\directeur}[1]{
  \def\@directeur{#1}
}

\def\@encadrant{encadrant}
\newcommand{\encadrant}[1]{
  \def\@encadrant{#1}
}
\def\@jurya{}{}{}
\newcommand{\jurya}[3]{
  \def\@jurya{#1	& #2	& #3\\}
}
\def\@juryb{}{}{}
\newcommand{\juryb}[3]{
  \def\@juryb{#1	& #2	& #3\\}
}
\def\@juryc{}{}{}
\newcommand{\juryc}[3]{
  \def\@juryc{#1	& #2	& #3\\}
}
\def\@juryd{}{}{}
\newcommand{\juryd}[3]{
  \def\@juryd{#1	& #2	& #3\\}
}
\def\@jurye{}{}{}
\newcommand{\jurye}[3]{
  \def\@jurye{#1	& #2	& #3\\}
}
\def\@juryf{}{}{}
\newcommand{\juryf}[3]{
  \def\@juryf{#1	& #2	& #3\\}
}
\def\@juryg{}{}{}
\newcommand{\juryg}[3]{
  \def\@juryg{#1	& #2	& #3\\}
}
\def\@juryh{}{}{}
\newcommand{\juryh}[3]{
  \def\@juryh{#1	& #2	& #3\\}
}
\def\@juryi{}{}{}
\newcommand{\juryi}[3]{
  \def\@juryi{#1	& #2	& #3\\}
}

\def\@rapporteura{}{}{}
\newcommand{\rapporteura}[2]{
  \def\@rapporteura{#1	& #2\\}
}
\def\@rapporteurb{}{}{}
\newcommand{\rapporteurb}[2]{
  \def\@rapporteurb{#1	& #2\\}
}

\makeatother

\newcommand\BackgroundPic{%
	\put(0,0){%
		\parbox[b][\paperheight]{\paperwidth}{%
			\includegraphics[width=\paperwidth]{fond.png}%
			\vfill
		}
	}
}

\newcommand\AbstractBackgroundPic{%
	\put(0,0){%
		\parbox[b][\paperheight]{\paperwidth}{%
			\includegraphics[width=\paperwidth]{fond_resume.png}%
			\vfill
		}
	}
}

\makeatletter
\newcommand{\pagedegarde}{
\newgeometry{top=1.1cm, bottom=1cm, left=0.8cm, right=0.8cm}
\AddToShipoutPictureBG*{\BackgroundPic}
  \begin{titlepage}
  \begin{minipage}{0.28\textwidth}
      \includegraphics[width=\textwidth]{ubl.png}
  \end{minipage}
  \hfill
  \begin{minipage}{0.25\textwidth}
      \includegraphics[width=\textwidth]{onera.png}
  \end{minipage}
  \hspace{1em}
  \begin{minipage}{0.2\textwidth}
      \includegraphics[width=\textwidth]{ubs.png}
  \end{minipage}
  \vspace*{0.065\paperheight}

  {\fontfamily{phv}\selectfont
  {\Huge \scshape{Thèse de doctorat de}}\\

  \vspace*{0.068\paperheight}

  {\Large L'UNIVERSITÉ BRETAGNE SUD}

  \medskip

  {\large \scshape{Comue Université Bretagne Loire}}

  \medskip

  {\large \scshape{École Doctorale N\degre601}}

  \medskip

  \emph{MAThématiques et Sciences et Technologies}

  \emph{de l'Information et de la Communication}

  \emph{Spécialité\,: \@specialite}

        \medskip
        ~~par\\
        \medskip
        {\Huge\hspace{2cm} \textbf{~~~\@author}}\\

        \medskip
        ~~intitulée\\
        \medskip

        {\LARGE~~~\@title}

        \medskip
        \vspace*{0.01\paperheight}

        \colorbox{white}{\centering\begin{minipage}{0.9\textwidth}
        \medskip
        Thèse présentée et soutenue à Palaiseau, le \@date,

        préparée à l'Institut de recherche en informatique et systèmes aléatoires (UMR 6074),

        et l'Office national d'études et de recherches aérospatiales.

        \medskip
        \textbf{Thèse n\up{o}\,: YYYY}

        \vspace*{0.03\paperheight}
        \begin{tabularx}{\textwidth}{llY}
          \multicolumn{3}{c}{\large \textbf{Rapporteurs avant soutenance\,:}}\\
          \\
          \@rapporteura
          \@rapporteurb
        \end{tabularx}

        \medskip

      	\begin{tabularx}{\textwidth}{lXl}
      		\multicolumn{3}{l}{\large \textbf{Composition du jury\,:}}\\
          \\
      		\@jurya
      		\@juryb
      		\@juryc
      		\@juryd
      		\@jurye
      		\@juryf
      		\@juryg
      		\@juryh
      		\@juryi
          \\
          \textbf{Directeur de thèse}\\
          \@directeur\\
          \textbf{Encadrant}\\
          \@encadrant\\
      	\end{tabularx}
     \end{minipage}}
  }
  \end{titlepage}
\newpage
\newgeometry{top=1.1cm, bottom=2cm, left=1cm, right=1cm}
\AddToShipoutPictureBG*{\AbstractBackgroundPic}
\begin{titlepage}
\begin{minipage}{0.28\textwidth}
    \includegraphics[width=\textwidth]{ubl.png}
\end{minipage}
\hfill
\begin{minipage}{0.25\textwidth}
    \includegraphics[width=\textwidth]{onera.png}
\end{minipage}
\hspace{1em}
\begin{minipage}{0.2\textwidth}
    \includegraphics[width=\textwidth]{ubs.png}
\end{minipage}
\definecolor{rulepink}{RGB}{244,173,179}
\definecolor{textpink}{RGB}{233,91,104}
\vspace*{2.3cm}

{\color{rulepink}\rule{\textwidth}{0.2cm}}

\bigskip

\begin{minipage}{\textwidth}
\large\fontfamily{phv}\selectfont
{\color{textpink} \textbf{Titre\,:}}~~\@title

\bigskip

\textbf{Mots clés\,:} apprentissage profond, télédétection, segmentation sémantique, cartographie, réseaux de neurones
\begin{multicols}{2}
\textbf{Résumé\,:}
L'observation de la Terre permet de modéliser et de comprendre son évolution. L'abondance d'images de télédétection aériennes et satellitaires nécessite la mise en œuvre de moyens d'analyse automatiques, capables d'interpréter ces données et de cartographier la surface du globe. Cette thèse traite de la conception, du déploiement et de la validation de stratégies d'apprentissage automatique, en particulier de réseaux de neurones convolutifs profonds, pour la compréhension d'images et la cartographie automatisée. Nous proposons des modèles pour l'interprétation d'images couleur, multispectrales et hyperspectrales, capables de prendre en compte les interactions spatiales entre entités géométriques et produisant des cartes d'une précision permettant la détection d'objets. Nous introduisons des architectures de fusion de données par apprentissage multi-modal et correction résiduelle afin de tirer parti des données ancillaires, comme les modèles numériques de terrain et les connaissances géographiques disponibles a priori. Enfin, nous étudions les capacités de généralisation de ces modèles dans des cas extrêmes de jeux de données limités ou massifs. Nous validons tout au long de cette thèse nos contributions sur de nombreux jeux de données aériens et satellitaires pour la classification des sols, l'étude de l'utilisation des terres, l'extraction de bâtiments et la détection de véhicules.

\end{multicols}
\end{minipage}
\vspace{4em}
%{\color{rulepink}\rule{\textwidth}{0.2cm}}

\bigskip

\begin{minipage}{\textwidth}
\large\fontfamily{phv}\selectfont
{\color{textpink} \textbf{Titre\,:}}~~\@titleen

\bigskip

\textbf{Mots clés\,:} deep learning, remote sensing, semantic segmentation, neural networks, mapping
\begin{multicols}{2}
\textbf{Résumé\,:}
Earth Observation allows us to modelize and understand the evolution of our planet. The profusion of aerial and satellite remote sensing images induce the need for automated tools able to semantize this raw data in order to map the Earth. This thesis studies the design, implementation and validation of machine learning strategies, specifically deep convolutional neural networks, for image understanding and automatic mapping. We introduce models for automated interpretation of color, multispectral and hyperspectral images, that are able to exploit spatial relationships between geometrical entities and producing high precision maps usable for object detection. We design data fusion architectures using multi-modal learning and residual correction that can leverage ancillary data, such as digital surface models and prior geographical knowledge. Finally, we study the generalization abilities of those networks for extreme cases of both limited and very large datasets. All along this work, we thoroughly validate our contributions on various aerial and satellite datasets for land cover and land use classification, building footprints extraction and vehicle detection.
\end{multicols}
\end{minipage}
\end{titlepage}

\restoregeometry
}
\makeatother
