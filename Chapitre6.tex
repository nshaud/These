%!TEX root = Manuscrit.tex
\chapter{Structure spatiale des cartes prédites}
\label{chap:structure}
%\citationChap{}{}
\minitoc

\chapsummary{%
Jusqu'à présent, nous nous sommes intéressé à la segmentation sémantique au travers le problème de classification dense. En effet, nous avons cherché à classifier les pixels indépendamment les uns des autres. Bien que les modèles aient pris en compte le contexte spatial, la fonction de coût ne s'intéressait qu'à réduire l'erreur moyenne sur les pixels. Toutefois, la compréhension de scènes nécessite bien d'extraire et de manipuler des concepts liés aux objets et à leurs instances, et non des pixels.

Ce chapitre s'intéresse ainsi aux extensions des modèles de réseaux de neurones permettant de jouer avec la structure spatiale des objets d'intérêt dans les images.

Dans un premier temps, nous nous intéressons à la possibilité de structurer \emph{a posteriori} les prédictions pixelliques issues des \gls{CNN} que nous avons entraîné afin d'analyser les images de télédétection sous l'angle des objets. En particulier, nous montrons que les cartes sémantiques obtenues par nos \gls{FCN} peuvent être aisément structurées pour permettre la détection et la reconnaissance individuelle d'instances de véhicules dans des images aériennes.

Dans un second temps, nous nous intéressons à des représentations alternatives de la vérité terrain et à des fonctions de coût auxiliaires permettant d'intégrer des notions spatiales dans l'optimisation des réseaux de neurones. Précisément, nous suggérons de compléter les annotations sémantiques par leur équivalent continu, obtenues par transformée en distance euclidienne. Cette représentation alternative peut ensuite servir à entraîner un modèle de régression, complémentaire au modèle de classification. En pratique, nous proposons d'entraîner un réseau multi-tâches réalisant en simultanée la régression des cartes de distances et la classification des pixels. Cette méthode permet d'intégrer une proportion de structure spatiale dans la fonction de coût et régularise ainsi les cartes sémantiques prédites.
}

\newpage

\section{Modèles objet}

\subsection{Régions et objets}

La détection et la reconnaissance de véhicules sont deux problèmes classiques en télédétection. De nombreux travaux ont étudié la détection automatique de véhicules dans des images \gls{THR} avec une large variété de méthodes, allant des descripteurs \gls{HOG} couplés à des \gls{SVM}~\cite{michel_local_2011,gleason_vehicle_2011,kamenetsky_aerial_2015} aux modèles 3D d'estimation de pose~\cite{janney_pose-invariant_2015} en passant par les modèles à parties déformables~\cite{randrianarivo_urban_2013} ou les mélanges de modèles invariants par rotation~\cite{randrianarivo_contextual_2016}. L'apprentissage profond et notamment les \gls{CNN} ont également été appliqués à cette tâche~\cite{chen_vehicle_2014}. Cependant, peu de travaux s'intéressent à réaliser simultanément détection et reconnaissance, en dépit de l'introduction de la base de données \gls{VEDAI} par~\citet{razakarivony_vehicle_2016} et d'une première approche utilisant des caractéristiques expertes pour classifier divers véhicules à partir d'images \gls{IRRVB}. Certains travaux plus anciens s'étaient intéressés à cette problématique sous la forme d'une segmentation multi-échelle et de l'utilisation de règles de logique floue~\cite{holt_object-based_2009}, puis d'analyse discriminante linéaire~\cite{eikvil_classification-based_2009}. En particulier, \citet{eikvil_classification-based_2009} montrent que segmenter l'image avant de réaliser la détection des véhicules permet d'éliminer de nombreux faux positifs.

Nos travaux s'inscrivent dans cette perspective, mais utilisent l'apprentissage profond afin de construire des représentations ne dépendant pas de l'expertise humaine. Nous essayons de réunir dans un même processus segmentation sémantique, détection d'objet et classification fine de véhicules en utilisant une approche par raffinements successifs. De fait, nous allons donc plus loin que la régression de boîtes englobantes car nous souhaitons aussi bien obtenir le type que la forme exacte des véhicules observés. Nous utilisons donc une approche \emph{segment-to-detect}, c'est-à-dire de segmentation pour la détection, la seconde étant un dérivé de la première.

\subsection{Modélisation objet par segmentation}

Notre méthode \emph{segment-before-detect} permet d'extraire et de classifier des véhicules à partir d'images \gls{THR} et consiste en trois étapes, illustrées dans la \cref{fig:pipeline}\,:
\begin{enumerate}
\item Segmentation sémantique et inférence des masques de véhicules au niveau pixel à partir d'un \gls{FCN},
\item Détection de véhicules par régression des enveloppes convexes des composantes connectées du masque de véhicules,
\item Classification des objets identifiés à l'aide d'un \gls{CNN}.
\end{enumerate}

\begin{figure}[t]
  \resizebox{\textwidth}{!}{%
  	\documentclass{standalone}
\usepackage{tikz}
\usepackage{ifthen}
\input{../CommandesPerso.tex}
\begin{document}
\begin{tikzpicture}[]

\node at (-5, 0) {\includegraphics[width=4cm]{/home/naudeber/Documents/Timeless/Publications/GEOBIA16/images/sliding_window_rgb}};
\node at (-5, -3) {\Large{Fenêtre glissante}};
\node at (-5, -3.75) {\Large{sur l'image RVB}};

\draw [ultra thick,->] (-2, 0) -- node[above] {\includegraphics[width=4cm]{/home/naudeber/Documents/Timeless/Manuscrit/Chapitre2/segnet.pdf}} node[below] {\large réseau profond} (1,0);
\node at (-0.5, -1) {\large de segmentation};

\node at (4, 0) {\includegraphics[width=4cm]{/home/naudeber/Documents/Timeless/Publications/GEOBIA16/images/potsdam_vehicle_0_pred}};
\node at (4, -3) {\Large Carte sémantique};

\draw [ultra thick,->] (6.5, 0) -- node[above] {\large extraction} node[below] {\large des véhicules} (9.5,0);

\node at (12, 0) {\includegraphics[width=4cm]{/home/naudeber/Documents/Timeless/Illustrations/potsdam_vehicle_0_gt_vehicles}};
\node at (12,-3) {\Large Masque des véhicules};

\draw [ultra thick,->] (14.5, 0) -- node[above] {\includegraphics[width=3.8cm]{/home/naudeber/Documents/Timeless/Illustrations/tikz/alexnet.pdf}} node[below] {\large réseau profond} (17.5,0);
\node at (16, -1) {\large classification};

\node at (20, 0) {\includegraphics[width=4cm]{/home/naudeber/Documents/Timeless/Publications/GEOBIA16/images/classif}};
\node at (20, -3) {\Large Véhicules classifiés};

\end{tikzpicture}
\end{document}

  }
  \caption{Illustration de la méthode \emph{segment-before-detect} pour la segmentation, détection et classification de véhicules.}
  \label{fig:pipeline}
\end{figure}

\subsection{Détection de petits objets}

Comme nous l'avons vu dans les sections précédentes, un réseau profond de type SegNet est suffisamment précis pour prédire des véhicules individuels tant que la résolution des images est suffisante. Dans ce cas, il suffit alors de réaliser une extraction des composantes connectées pour obtenir les instances individuelles des véhicules présents dans l'image. Pour chaque composante connectée, le calcul de la boîte englobante l'entourant est alors immédiat.

Cependant, les prédictions issues de SegNet sont susceptibles d'être bruitées. En particulier, les \gls{CNN} appliqués à l'observation de la Terre tendent à produire des transitions inter-classes imprécises~\cite{marmanis_classification_2017}. Par conséquent, nous éliminons une grande partie des faux positifs et séparons les véhicules susceptibles d'avoir été prédits comme appartenant à la même composante connexe en appliquant une ouverture morphologique au masque sémantique obtenu par SegNet. Nous éliminons ensuite les objets dont la surface est inférieure à un seuil strict pour supprimer les fausses alarmes dûes aux erreurs de segmentation, comme les bouches de ventilation sur les toits ou certains éléments du mobilier urbain. Bien que simple, cette approche suffit à obtenir d'excellents résultats en détection de véhicules.

\subsection{Reconnaissance de véhicules par \gls{CNN}}

En admettant que nous disposons de véhicules potentiels, la tâche principale reste d'en identifier le type. Ainsi, nous cherchons à déterminer si le véhicule est une voiture, un camion, une camionette, etc. Il s'agit d'un problème classique de reconnaissance d'objet que les \gls{CNN} peuvent aisément résoudre. Nous appliquons donc l'approche classique consistant à spécialiser~\cite{nogueira_towards_2016,zhou_deep_2016} un modèle de \gls{CNN} pré-entraîné sur la base de données ImageNet~\cite{russakovsky_imagenet_2015} pour la reconnaissance de véhicules.

Nous comparons en particulier les modèles les plus utilisés dans la litérature pour la classification de petites images ($\simeq$30 $\times$ 30)\,: LeNet~\cite{lecun_gradient-based_1998}, AlexNet~\cite{krizhevsky_imagenet_2012} et VGG-16~\cite{simonyan_very_2014}.

Notre objectif est d'entraîner ce classifieur sur une grande base de données de véhicules et de l'appliquer sur des données issues d'une scène spécifique. Nous risquons donc d'être confrontés au problème de surapprentissage. En effet, nous cherchons à transférer des connaissances d'un jeu à de données à un autre. Pour améliorer la généralisabilité du modèle, deux techniques peuvent être employées\,: l'adaptation de domaine et l'augmentation de données. L'adaptation de domaine cherche à minimiser les différence entre le jeu de donnée d'apprentissage et celui d'inférence. L'augmentation de données cherche quant à elle à générer de nouvelles images d'entraînement synthétiques pour améliorer la robustesse du classifieur.

Nous proposons ainsi de normaliser l'ensemble des véhicules pour que ceux-ci présentent le même azimuth, c'est-à-dire que toutes les images présentent un véhicule dont la direction principale est horizontale. Durant l'apprentissage, nous utilisons les boîtes englobantes pour estimer la direction principale du véhicule puis nous appliquons une rotation pour l'alignement. Durant l'inférence, nous utilisons la composante connectée du masque correspondant au véhicule pour réaliser cette opération.

Enfin, nous augmentons le jeu de données d'apprentissage en appliquant diverses opérations géométriques pour augmenter la variété des échantillons\,: translations ($\pm$\SI{10}{\px}), zooms (jusqu'à 1,25$\times$), rotations (90\degre, 180\degre and 270\degre) et symétries axiales, comme illustrées par la~\cref{fig:augmented_car}. Lorsque la stratégie de réalignement est utilisée, seule la rotation à 180\degre est appliquée.

\begin{figure}[t]
	\begin{subfigure}{0.135\textwidth}
    	\includegraphics[width=\textwidth]{images/1}
        \caption{}
    \end{subfigure}
    \begin{subfigure}{0.135\textwidth}
    	\includegraphics[width=\textwidth]{images/2}
        \caption{}
    \end{subfigure}
    \begin{subfigure}{0.1375\textwidth}
    	\includegraphics[width=\textwidth]{images/3}
        \caption{}
    \end{subfigure}
    \begin{subfigure}{0.135\textwidth}
    	\includegraphics[width=\textwidth]{images/4}
        \caption{}
    \end{subfigure}
    \begin{subfigure}{0.135\textwidth}
    	\includegraphics[width=\textwidth]{images/5}
        \caption{}
    \end{subfigure}
    \begin{subfigure}{0.135\textwidth}
    	\includegraphics[width=\textwidth]{images/6}
        \caption{}
    \end{subfigure}
    \begin{subfigure}{0.135\textwidth}
    	\includegraphics[width=\textwidth]{images/7}
        \caption{}
    \end{subfigure}\vspace{-12pt}
	\caption{Augmentation de données sur un échantillon de véhicules de la base \gls{VEDAI}. (\textbf{a}) originale; (\textbf{b}) rotation à 90\degre; (\textbf{c}) rotation à 180\degre; (\textbf{d}) symétrie horizontale; (\textbf{e}) symétrie verticale; (\textbf{f}) zoom; (\textbf{g}) translation.}
    \label{fig:augmented_car}
\end{figure}

\section{Expériences}
\unskip
\subsection{Jeux de données}
\unskip
\subsubsection{VEDAI}

La base de données \gls{VEDAI}~\cite{razakarivony_vehicle_2016} contient 1 268 images \gls{RVB} ($1024 \times 1024$ px) et autant d'images infrarouges associées (IR) à une résolution de \SI{12,5}{\meter/\px}. Pour chaque tuile, les annotations fournies contiennent le type de chaque véhicule présent, les coordonnées de son centroïde et les coordonnées du quadrilatère englobant. \gls{VEDAI} est utilisé pour entraîner le \gls{CNN} que nous utilisons pour la reconnaissance de véhicules, qui sera appliqué en inférence sur le jeu de données \gls{ISPRS}  Potsdam. Les résultats sur \gls{VEDAI} sont obtenus par validation croisée en utilisant 2/3 des images pour l'entraînement et 1/3 pour l'inférence.

\subsubsection{ISPRS Potsdam}

Nous utilisons pour valider notre méthode le jeu de données \gls{ISPRS} Potsdam sur lequel nous avons manuellement annoté les véhicules dans quatre sous-catégories\,: voitures, vans, camions et pick-ups. Les véhicules ayant été initiallement annotés dans la classe de rejet dans la vérité terrain originale ne sont pas pris en compte. Comme indiqué dans le~\cref{tab:vehicle_counts}, le jeu de données est largement dominé par les voitures (94\% des véhicules).

Nous entraînons un SegNet pour la segmentation sémantique sur ce jeu de données et nous appliquons le \gls{CNN} entraîné sur \gls{VEDAI} pour la reconnaissance de véhicules. Les résultats sont obtenus par validation croisée utilisant 18 tuiles pour l'entraînement et 6 tuiles pour la validation. La résolution spatiale est interpolée à \SI{12,5}{\centi\meter/\px} pour correspondre avec celle de \gls{VEDAI}.

\begin{figure}[t]
\centering
	\begin{subfigure}[]{0.45\textwidth}
    	\includegraphics[angle=90,width=\textwidth]{images/potsdam_illu}
   \caption{}
%        \label{fig:datasets_potsdam}
    \end{subfigure}
%     \begin{subfigure}{0.32\textwidth}
%     	\includegraphics[angle=90,width=\textwidth]{images/vaihingen_illu}
%         \caption{ISPRS Vaihingen dataset (IRRG)}
%         \label{fig:datasets_vaihingen}
%     \end{subfigure}
    \begin{subfigure}[]{0.45\textwidth}
    	\includegraphics[angle=90,width=\textwidth]{images/christchurch_illu}
\caption{}
%        \label{fig:datasets_christchurch}
    \end{subfigure}\vspace{-12pt}
    \caption{(\textbf{a}) ISPRS Potsdam; (\textbf{b})~NZAM/ONERA Christchurch.}
    \label{fig:datasets}
\end{figure}
\unskip
\begin{table}[t]
	\setlength\tabcolsep{2.5pt}
    \caption{Nombre de véhicules par classe dans les différents jeux de données.}
    \label{tab:vehicle_counts}
	\begin{tabularx}{\textwidth}{c c c c c c c c cc}
    \toprule
    \textbf{Jeu de donnée} & \textbf{Voiture} & \textbf{Camion} & \textbf{Van} & \textbf{Pick-up} & \textbf{Bateau} & \textbf{Camping-car} & \textbf{Autres} & \textbf{Avion} & \textbf{Tracteur}\\
    \midrule
    VEDAI & 1340 &  300 & 100 & 950 & 170 & 390 & 200 & 47 & 190\\
    NZAM/ONERA Christchurch & 2267 & 73 & 120 & 90 & - & - & - & - & -\\
    ISPRS Potsdam & 1990 & 33 & 181 & 40 & - & - & - & - & -\\
    \bottomrule
    \end{tabularx}
\end{table}

\subsubsection{NZAM/ONERA Christchurch}

Comme pour Potsdam, nous étendons les annotations du jeu de données NZAM/ONERA Christchurch aux voitures, camions, vans et pick-ups. À nouveau, le jeu de données est dominé par la classe voiture~\cref{tab:vehicle_counts}. À nouveau, nous entraînons un SegNet pour la segmentation sémantique sur ce jeu de données et nous appliquons le \gls{CNN} entraîné sur \gls{VEDAI} pour la reconnaissance de véhicules. Les résultats sont obtenus par validation croisée utilisant 3 tuiles pour l'entraînement et 1 tuile pour la validation. La résolution spatiale est interpolée à \SI{12,5}{\centi\meter/\px} pour correspondre avec celle de \gls{VEDAI}.

\subsection{Semantic Segmentation\label{sec:semantic_seg}}
\begin{figure}[t]
	\centering
	\begin{subfigure}{0.3\textwidth}
    	\includegraphics[width=\textwidth]{images/christchurch_extract_rgb}
        %\caption*{RGB (Christchurch)}
    \end{subfigure}
    \begin{subfigure}{0.3\textwidth}
    	\includegraphics[width=\textwidth]{images/christchurch_extract_gt}
        %\caption*{Ground truth (Christchurch)}
    \end{subfigure}
    \begin{subfigure}{0.3\textwidth}
    	\includegraphics[width=\textwidth]{images/christchurch_extract_pred}
        %\caption*{SegNet prediction (Christchurch)}
    \end{subfigure}\\
    	\begin{subfigure}{0.3\textwidth}
    	\includegraphics[width=\textwidth]{images/parking_zoom_rgb}
        \caption{}
    \end{subfigure}
    \begin{subfigure}{0.3\textwidth}
    	\includegraphics[width=\textwidth]{images/parking_zoom_gt}
        \caption{}
    \end{subfigure}
    \begin{subfigure}{0.3\textwidth}
    	\includegraphics[width=\textwidth]{images/parking_zoom_pred}
        \caption{}
    \end{subfigure}\vspace{-12pt}

    \caption{Résultats de segmentation (première ligne\,: NZAM/ONERA Christchurch, deuxième ligne\,: ISPRS Potsdam). \textbf{(a)} image \gls{RVB}, \textbf{(b)} vérité terrain, \textbf{(c}) prédiction de SegNet.\\\isprslegende}
    \label{fig:segmentations}
\end{figure}
\subsubsection{ISPRS Potsdam}

Nous détaillons dans le~\cref{table:potsdam_seg} les scores $F_1$ et l'exactitude globale du SegNet entraîné sur Potsdam. Il est important de noter que les résultats obtenus à résolution de \SI{12,5}{\centi\meter/\px} sont similaires à ceux obtenues à \SI{5}{\centi\meter/\px}. Un exemple qualitatif de segmentation est illustré dans la~\cref{fig:segmentations}.

\begin{table}[t]
\centering
	\caption{Résultats de segmentation sémantique sur le jeu de données \gls{ISPRS} Potsdam (scores $F_1$ et exactitude globale).}
    \label{table:potsdam_seg}
    %\setlength{\tabcolsep}{2pt}
    \scalebox{0.8}[0.8]{
	\begin{tabular}{cccccccc}
 	\toprule
    \textbf{Jeu de données }&\textbf{ Méthode} & \textbf{Routes} & \textbf{Bâtiments} & \textbf{Vég. basse} & \textbf{Arbres} & \textbf{Véhicules} & \textbf{Exactitude}\\
    \midrule
    Validation 12.5cm/px & SegNet RGB & 92.4\%  $\pm$ 0.6 & 95.8\%  $\pm$ 1.9 & 85.8\%  $\pm$ 1.3 & 83.0\%  $\pm$ 2.1 & 95.7\%  $\pm$ 0.3 & 90.6\%  $\pm$ 0.6\\
	%\midrule
    \midrule
    \multirow{3}{*}{\parbox{1.4cm}{Test 5cm/px}} & SegNet IRRG & 92.4\% & 95.8\% & \textbf{86.7\%} & 87.4\% & \textbf{95.1\%} & \textbf{90.0\%}\\
    & \glsname{FCN} + \glsname{CRF}~\cite{sherrah_fully_2016} & 91.8\% & 95.9\% & 86.3\% & 87.7\% & 89.2\% & 89.7\%\\
    & ResNet-101 [CASIA] & \textbf{92.8\%} & \textbf{96.9\%} & 86.0\% & \textbf{88.2\%} & 94.2\% & 89.6\%\\
    \bottomrule
    \end{tabular}}
\end{table}

\subsubsection{NZAM/ONERA Christchurch}

Le jeu de données NZAM/ONERA Christchurch contenant seulement des boîtes englobantes, possiblement recouvrantes, pour les arbres, les bâtiments et les véhicules, nous transformons ces annotations en cartes sémantiques. Pour ce faire, nous définissons quatre classes d'intérêt\,: arrière-plan, bâtiments, végétation et véhicules. Nous construisons la vérité terrain dense en étiquetant d'abord les pixels appartenant à une boîte englobante de bâtiment, puis ceux appartenant à des véhicules et finalement à la végétation. Les pixels restants sont étiquetés comme arrière-plan. Cet ordre permet de prendre en compte la présence de certains véhicules sur des parkings en toit de bâtiment et la présence de végétation arborescente pouvant masquer certaines voitures. Pour prendre en compte l'incertitude sur les boîtes englobantes, les pixels appartenant à un disque de rayon 5px autour des frontières sont ignorés.

Comme illustré dans la~\cref{table:christchurch_seg}, un SegNet entraîné sur NZAM/ONERA Christchurch atteint 61.9\% d'exactitude pixel à pixel sur les véhicules, ce qui est acceptable compte-tenu de la grossiereté des annotations en comparaison de celles de Potsdam. Ce constat est intéressant dans la mesure où il prouve qu'il est possible d'apprendre des modèles de segmentation sémantique, même en présence de simples boîtes englobantes originellement prévues pour entraîner des détecteurs. L'inférence sur une tuile de Christchurch nécessite environ 120 secondes sur un \gls{GPU} NVIDIA Tesla K20c. Un exemple de segmentation est illustré dans la~\cref{fig:segmentations}.

\begin{table}[t]
\centering
	\caption{Résultats de segmentation sémantique sur NZAM/ONERA Christchurch (scores $F_1$ et exactitude globale).}
    \label{table:christchurch_seg}
	\begin{tabular}{cccccc}
 	\toprule
    \textbf{} & \textbf{Arrière-plan} & \textbf{Bâtiments} & \textbf{Végétation} & \textbf{Véhicules} & \textbf{Exactitude}\\
    \midrule
    \glsname{RVB} & 75.6\% $\pm$ 8.9 & 91.7\% $\pm$ 1.3 & 55.2\%  $\pm$ 11.6 & 61.9\%  $\pm$ 2.4 & 84.4\%  $\pm$ 2.6\\
    \bottomrule
    \end{tabular}
\end{table}

\subsection{Détection de véhicules}
\begin{figure}[t]
\centering
	\begin{subfigure}{0.3\textwidth}
    	\includegraphics[width=\textwidth]{images/christchurch_draw1}
    	\caption{}
    \end{subfigure}
    \begin{subfigure}{0.3\textwidth}
    	\includegraphics[width=\textwidth]{images/christchurch_draw2}
    	\caption{}
    \end{subfigure}\begin{subfigure}{0.3\textwidth}
    	\includegraphics[width=\textwidth]{images/potsdam_draw1}
    	\caption{}
    \end{subfigure}\vspace{-12pt}
	\caption{Exemples de détections sur Potsdam et Christchurch (vrais positifs en vert, faux positifs en rouge et vérité terrain en bleu). (\textbf{a}) Christchurch; (\textbf{b}) Christchurch; (\textbf{c}) Potsdam.}
    \label{fig:detection_samples}
\end{figure}

Nous appliquons sur les deux jeux de données une ouverture morphologique d'un rayon de 3px ($\simeq$ \SI{35}{\centi\meter} d'incertitude sur les formes de véhicules prédites) pour isoler des véhicules ayant pu être prédits dans la même composante connexe. Nous supprimons également les composantes connexes couvrant moins de 100px. Nous extrayons ensuite les composantes connexes dans le masque des véhicules et nous calculons la boîte englobante de chaque composante. Pour le jeu de données \gls{ISPRS} Potsdam, comme la vérité terrain initiale est constituée d'annotations pixelliques denses, nous avons regressé une boîte englobante pour chaque composante connectée en corrigeant manuellement les rares erreurs.

Nous suivons les pratiques habituelles en détection d'objet~\cite{everingham_pascal_2014} et nous définissons un vrai positif comme une boîte englobante prédite dont l'intersection sur union avec une boîte englobante de la vérité terrain dépasse 0,5. Si plusieurs prédictions existent pour le même véhicule, nous conservons celle avec la plus haute valeur d'\gls{IsU}. Les autres prédictions sont considérées comme des fausses alarmes. Sur le jeu de données NZAM/ONERA Christchurch, nous comparons nos résultats sur la même tuile que celle choisie par~\citet{randrianarivo_contextual_2016}, qui ont appliqué un \emph{Discriminatively trained Model Mixture} (DtMM) contenant cinq modèles, un pour chaque orientation principale.

Pour évaluer l'effet de l'ouverture morphologique sur les performances en segmentation d'instance, nous détaillons dans le~\cref{table:morpho_results} la moyenne de l'\gls{IsU} calculée sur les instances de véhicules et les scores précision/rappel finaux en fonction des divers pré-traitements appliqués. Ceci montre que la simple ouverture morphologique et l'élimination des véhicules candidats dont la surface est inférieure à 100px permet de grandement améliorer les performances en éliminant les faux positifs. Ceci est particulièrement vrai pour le jeu de données NZAM/ONERA dans lequel les annotations grossières conduisent à des cartes sémantiques imprécises après inférence. Le processus complet atteint un score \gls{IsU} de plus de 74\% sur Potsdam et plus de 70\% sur Christchurch.

Finalement, nous indiquons les résultats de détection dans le~\cref{table:detection_results}. Sur NZAM/ONERA Christchurch, notre méthode \textit{segment-before-detect} obtient des résultats significativement supérieurs aux approches par mélange de modèles et \gls{HOG}+\gls{SVM}. Bien qu'aucune autre méthode de détection de véhicules n'ait été appliquée sur Potsdam, nous indiquons également nos scores de précision/rappel sur ce jeu de données. Des exemples qualitatifs de détection sont présentés dans la~\cref{fig:detection_samples}.

\begin{table}[t]
\centering
  \caption{Segmentation d'instance et détection de véhicules pour différents pré-traitements morphologiques (\gls{IsU} moyen, précision et rappel).}
  \label{table:morpho_results}
  \begin{tabular}{ccccc}
  \toprule
  \textbf{Jeu de données} & \textbf{Pré-traitement} & \textbf{\gls{IsU} moyen} & \textbf{Précision} & \textbf{Rappel}\\
  \midrule
  %\multirow{3}{*}{Potsdam} & AlexNet & 81\% & 42\% & \textbf{50\%} & \textbf{50\%} & 75\% & 56\%\\
  %& VGG & 83\% & 55\% & \textbf{50\%} & 33\% & 77\% & 55\%\\
  %& AlexNet+R & \textbf{99\%} & \textbf{66\%} & 25\% & 0\% & \textbf{83\%} & 48\%\\
  \multirow{3}{*}{NZAM/ONERA Christchurch} & $\emptyset$ & 60.0\% & 0.597 & 0.797\\
  & Ouverture & 69.8\% & 0.817 & 0.791\\
  & Ouverture + retrait des petits objets & 70.7\% & 0.833 & 0.791\\
  \midrule
  \multirow{3}{*}{ISPRS Potsdam} & $\emptyset$ & 70.1\% & 0.748 & 0.842\\
  & Ouverture & 73.3\% & 0.866 & 0.842\\
  & Ouverture + retrait des petits objets & 74.2\% & 0.907 & 0.841\\
  \bottomrule
  \end{tabular}
\end{table}
\unskip
\begin{table}[t]
\centering
  \caption{Résultats de détection de véhicules sur Potsdam et Christchurch.}
  \label{table:detection_results}
  \begin{tabular}{cccc}
  \toprule
  \textbf{Jeu de données} & \textbf{Méthode} & \textbf{Précision} & \textbf{Rappel}\\
  \midrule
  %\multirow{3}{*}{Potsdam} & AlexNet & 81\% & 42\% & \textbf{50\%} & \textbf{50\%} & 75\% & 56\%\\
  %& VGG & 83\% & 55\% & \textbf{50\%} & 33\% & 77\% & 55\%\\
  %& AlexNet+R & \textbf{99\%} & \textbf{66\%} & 25\% & 0\% & \textbf{83\%} & 48\%\\
  \multirow{3}{*}{NZAM/ONERA Christchurch} & HOG+SVM~\cite{michel_local_2011} & 0.402 & 0.398\\
  & \emph{DtMM} (5 modèles)~\cite{randrianarivo_contextual_2016} & 0.743 & 0.737\\
  & \emph{Segment-before-detect} & \textbf{0.833} & \textbf{0.791}\\
  \midrule
  ISPRS Potsdam & \emph{Segment-before-detect} & 0.907 & 0.841\\
  \bottomrule
  \end{tabular}
\end{table}

Christchurch est un jeu de données plus complexe pour deux raisons. Tout d'abord, la densité de véhicules y est nettement plus élevée que pour Potsdam, la ville comprenant de nombreux véhicules resserrés sur des surfaces réduites. Ensuite, les annotations grossières empêchent le \gls{FCN} de prédire correctement les frontières des objets constituant la scène, ce qui conduit à des masques de véhicules imprécis (cf. \cref{fig:segmentations}, l'\gls{IsU} moyen sur Christchurch atteint 66.6\% contre plus de 80\% pour Potsdam). Cette combinaison rend le problème d'extraction des instances de véhicules plus difficile, mais reste toutefois à la portée de notre approche. Cependant, les boîtes englobantes obtenues tendent à couvrir plusieurs véhicules .
En dépit de cela, il est intéressant de constater que les résultats de segmentation sémantique sont relativement corrects, alors même que les annotations initiales étaient prévues pour la détection d'objet. Ainsi, la segmentation peut être utilisée comme étape intermédiaire pour des tâches de détection, pour lesquelles l'état de l'art nécessite des approches sophistiquées et complexes. L'étape d'extraction de composantes connectées pourrait en outre être améliorée en utilisant une extraction de boîte englobante plus robuste, soit par des approches morphologiques comme un \emph{watershed} appliqué aux cartes de probabilités~\cite{beucher_morphological_1993}, soit en intégrant la prédiction d'instance au sein du réseau~\cite{dai_instance-aware_2015,he_mask_2017}.

\subsection{Classification de véhicules}

Afin de classifier les véhicules identifiés, nous souhaitons utiliser un \gls{CNN}. Nous comparons trois architectures de complexité croissante\,: LeNet~\cite{lecun_gradient-based_1998}, AlexNet~\cite{krizhevsky_imagenet_2012} et VGG-16~\cite{simonyan_very_2014}.

 LeNet-5 est un petit \gls{CNN} que nous entraînons à partir d'une initialisation aléatoire sur les véhicules de \gls{VEDAI} en utilisant des images de dimensions $32\times32$. AlexNet et VGG-16 sont deux réseaux ayant gagné la compétition \gls{ILSVRC} en 2012 et 2014. Des expériences préliminaires montrent que l'utilisation des poids de ces réseaux pré-entraînés sur ImageNet permet d'augmenter de 10\% l'exactitude des modèles, ce qui est cohérent avec la litérature~\cite{nogueira_towards_2016,penatti_deep_2015}. Par conséquent, ces \gls{CNN} seront simplement spécialisés sur des images de véhicules à résolution $224 \times 224$ et $227 \ times 227$. Ces tailles d'images nous permettent de conserver les poids pré-entraînés des couches entièrement connectées. En pratique, les véhicules issues des images aériennes auront une taille d'environ $25\times 25$. En pratique, nous extrayons des imagettes centrées sur ces véhicules incluant un contexte spatial de \SI{16}{\px} dans toutes les directions, cette approche ayant donné les meilleurs résultats. Un contexte plus large risquerait d'inclure d'autres véhicules dans la même imagette tandis qu'un contexte plus restreint diminue la quantité d'information disponible pour la classification. Les imagettes sont ensuites redimensionnées par interpolation bilinéaire selon leur plus grande dimension pour correspondre à la taille attendue par le \gls{CNN}, la plus petite dimension étant remplie par du bruit blanc.

Les modèles sont entraînés (ou spécialisés) durant 20 passes sur le jeu de données à l'aide d'une descente de gradient stochastique avec moment. Nous utilisons des mini-lots de 128 échantillons pour AlexNet et LeNet et de 32 pour VGG-16 compte-tenu de l'espace mémoire requis. Le taux d'apprentissage est initialement fixé à 0.01 et est divisé par 10 après 75\% de l'entraînement. Pour les modèles soumis au \emph{fine-tuning}, nous ré-entraînons le réseau dans son intégralité à l'exception de la dernière couche, qui est initialisée aléatoirement et est entraînée avec un taux d'apprentissage 10 fois supérieur. Nous appliquons également du \emph{Dropout}~\cite{srivastava_dropout:_2014} aux couches entièrement connectées pour limiter le surapprentissage.

Sans surprise, les performances des \gls{CNN} sur \gls{VEDAI} sont proportionnelles à leurs résultats sur ImageNet, comme montré dans le~\cref{table:cnn_benchmark}. Toutefois, le modèle le plus complexe (VGG-16) n'améliore que légèrement les résultats de classification comparé à l'important surcoût calculatoire qu'il engendre. En pratique, il pourrait être possible d'utiliser n'importe quel \gls{CNN}, incluant les ResNet~\cite{he_deep_2016}. Toutefois, nous nous contentons du modèle AlexNet qui offre un rapport performance/temps d'exécution acceptable.

Le \cref{table:da_benchmark} détaille les résultats de classification de véhicules sur \gls{VEDAI} en utilisant plusieurs stratégies de pré-traitement des données. L'augmentation de données par transformations géométriques (noté \emph{AD}) permet d'améliorer la robustesse et la généralisabilité du modèle. Le réalignement \emph{R} permet également d'améliorer les résultats et de rendre le réseau plus robuste en éliminant la nécessité d'apprendre une classification équivariante par rotation. La combinaison de ces deux stratégies permettant d'obtenir les meilleurs résultats, nous les appliquons donc toutes deux sur les jeux de données \gls{ISPRS} Potsdam et NZAM/ONERA Christchurch.

\begin{table}[t]
\centering
	\caption{Résultats de classification de plusieurs \gls{CNN} sur \gls{VEDAI} (en \%).}
    \label{table:cnn_benchmark}
    \setlength\tabcolsep{3pt}
	\begin{tabular}{cccccccccccc}
    \toprule
    \textbf{Modèle} &  \textbf{Voiture} & \textbf{Camion} & \textbf{Bateau} & \textbf{Tracteur} & \textbf{Camping-car} & \textbf{Van} & \textbf{Pick-up} & \textbf{Avion} & \textbf{Autres} & \textbf{OA} & \textbf{Time (ms)}\\
    \midrule
  LeNet & 74.3 & 54.4 & 31.0 & 61.1 & 85.9 & 38.3 & 67.7 & 13.0 & 47.5 & 66.3 $\pm$ 1.7  & \textbf{2.1}\\
  AlexNet & \textbf{91.0} & 84.8 & 81.4 & 83.3 & 98.0 & \textbf{71.1} & 85.2 & 91.4 & \textbf{77.8} & 87.5 $\pm$ 1.5 & 5.7\\
  VGG-16 & 90.2 & \textbf{86.9} & \textbf{86.9} & \textbf{86.5} & \textbf{99.6} & \textbf{71.1}
& \textbf{91.4} & \textbf{100.0} & 77.2 & \textbf{89.7} $\pm$ 1.5 & 31.7\\
    \bottomrule
    \end{tabular}
\end{table}
\unskip
\begin{table}[t]
\centering
	\caption{Résultats de classification d'AlexNet sur \gls{VEDAI} avec plusieurs pré-traitements (en \%).}
    \label{table:da_benchmark}Modèle
   % \setlength\tabcolsep{4pt}
   \scalebox{0.9}[0.9]{
	\begin{tabular}{cccccccccccc}
    \toprule
    \textbf{Pré-traitement} &  \textbf{Voiture} & \textbf{Camion} & \textbf{Bateau} & \textbf{Tracteur} & \textbf{Camping-car} & \textbf{Van} & \textbf{Pick-up} & \textbf{Avion} & \textbf{Autres} & \textbf{OA} & \textbf{AA}\\
    \midrule
    %Standard & 90.4\% $\pm$ 3.9 & 66.7\% $\pm$ 10.0 & 80.4\% $\pm$ 14.8 & 89.5\% $\pm$ 9.1 & 96.6\% $\pm$ 1.5 & 63.3\% $\pm$ 15.3 & 78.7\% $\pm$ 2.8 & 92.6\% $\pm$ 12.8 & 75.0\% $\pm$ 5.0 & 83.9\% $\pm$ 2.7 & 81.5\% $\pm$ 1.9\\
    %DA & 88.2\% $\pm$ 0.6 & 82.2 $\pm$ 3.1 & 78.4\% $\pm$ 7.3 & 82.5\% $\pm$ 5.0 & 97.4\% $\pm$ 3.6 & 63.3\% $\pm$ 4.7 & 85.1\% $\pm$ 4.8 & 66.7 $\pm$ 47.1 & 73.3\% $\pm$ 8.5 & 85.6\% $\pm$ 1.4 & 77.3\% $\pm$ 8.7\\
    %R & 87.9 $\pm$ 1.7 & 71.1\% $\pm$ 1.6 & 86.3\% $\pm$ 7.3 & 84.2\% $\pm$ 7.4 & 97.4\% $\pm$ 2.1 & 73.3\% $\pm$ 4.7 & 87.2\% $\pm$ 3.1 & 100.0\% $\pm$ 0.0 & 75.0\% $\pm$ 7.1 & 86.1\% $\pm$ 0.9 & 84.7\% $\pm$ 1.7\\
    %DA + R & 91.4\% $\pm$ 2.4 & 85.6\% $\pm$ 8.3 & 88.2\% $\pm$ 8.3 & 87.6\% $\pm$ 6.4 & 97.4\% $\pm$ 2.0 & 70.0\% $\pm$ 8.1 & 87.2\% $\pm$ 1.5  & 100.0\% $\pm$ 0.0 & 81.7\% $\pm$ 8.5 & 89.0\% $\pm$ 0.5 & 87.7\% $\pm$ 1.5\\
    $\emptyset$ & 90.4 & 66.7 & 80.4 & \textbf{89.5} & 96.6 & 63.3 & 78.7 & 92.6 & 75.0 & 83.9 \footnotesize $\pm$ 2.7 & 81.5 \footnotesize $\pm$ 1.9\\
    AD & 88.2 & 82.2 & 78.4 & 82.5 & \textbf{97.4} & 63.3 & 85.1 & 66.7 & 73.3 & 85.6 \footnotesize $\pm$ 1.4 & 77.3 \footnotesize $\pm$ 8.7\\
    R & 87.9 & 71.1 & 86.3 & 84.2 & \textbf{97.4} & 73.3 & \textbf{87.2} & \textbf{100.0} & 75.0 & 86.1 \footnotesize $\pm$ 0.9 & 84.7 \footnotesize $\pm$ 1.7\\
    AD + R & \textbf{91.4} & \textbf{85.6} & \textbf{88.2} & 87.6 & \textbf{97.4} & \textbf{70.0} & \textbf{87.2} & \textbf{100.0} & \textbf{81.7} & \textbf{89.0} \footnotesize $\pm$ 0.5 & \textbf{87.7} \footnotesize $\pm$ 1.5\\
    \bottomrule
    \end{tabular}}
    \begin{tabular}{ccc}
\multicolumn{1}{c}{\footnotesize  DA = data augmentation, R = renormalization.}
\end{tabular}
\end{table}

\subsection{Transfert de connaissances pour la classification de véhicules}
\begin{figure}[t]
\centering
   \begin{subfigure}{0.24\textwidth}
     \includegraphics[width=\textwidth]{images/negative_example1}
     \caption{}
   \end{subfigure}
  \begin{subfigure}{0.24\textwidth}
    \includegraphics[width=\textwidth]{images/negative_example2}
    \caption{}
  \end{subfigure}
  \begin{subfigure}{0.24\textwidth}
    \includegraphics[width=\textwidth]{images/negative_example3}
    \caption{}
  \end{subfigure}
  \begin{subfigure}{0.24\textwidth}
    \includegraphics[width=\textwidth]{images/negative_example4}
    \caption{}
  \end{subfigure}\vspace{-12pt}

  \caption{Segmentation réussies mais mauvaises classification sur Potsdam. (\textbf{a}) van prédit comme camion; (\textbf{b}) van prédit comme camion; (\textbf{c}) voiture (SUV) prédit comme van; (\textbf{d}) pick-up prédit comme van.}
  \label{fig:negative_examples}
\end{figure}
\unskip
\begin{figure}[t]
\centering
  \begin{subfigure}{0.24\textwidth}
    \includegraphics[width=\textwidth]{images/positive_example1}
    \caption{}
  \end{subfigure}
%   \begin{subfigure}{0.24\textwidth}
%     \includegraphics[width=\textwidth]{images/positive_example2}
%     \caption{Van}
%   \end{subfigure}
  \begin{subfigure}{0.24\textwidth}
    \includegraphics[width=\textwidth]{images/positive_example3}
    \caption{}
  \end{subfigure}
  \begin{subfigure}{0.24\textwidth}
    \includegraphics[width=\textwidth]{images/positive_example4}
    \caption{}
  \end{subfigure}
  \begin{subfigure}{0.24\textwidth}
    \includegraphics[width=\textwidth]{images/positive_example5}
    \caption{}
  \end{subfigure}\vspace{-12pt}

%   \begin{subfigure}{0.24\textwidth}
%     \includegraphics[width=\textwidth]{images/positive_example6}
%     \caption{Car}
%   \end{subfigure}
  \caption{Segmentation et classification réussies sur Potsdam. (\textbf{a}) pick-up; (\textbf{b})~van; (\textbf{c}) camion; (\textbf{d}) voiture.}
  \label{fig:positive_examples}
\end{figure}

À ce stade, nous disposons d'un détecteur de véhicules efficace pour Potsdam et Christchurch et d'un classifieur entraîné sur \gls{VEDAI}. Le~\Cref{table:results_classif} détaille donc les résultats de classification du \gls{CNN} entraîné sur \gls{VEDAI} et appliqué aux véhicules de Potsdam et Christchurch. Les résultats sont agrégés par validation croisée sur les mêmes sous-division des jeux de données que pour les résultats en segmentation sémantique. Les jeux de données étant dominés par les voitures, nous indiquons également les résultats qui seraient obtenus par une ligne de base correspondant à un classifieur renvoyant systématiquement ``voiture''. Ce classifieur constant serait correct 94\% du temps, mais serait incapable de prédire autre chose qu'une voiture. L'exactitude globale du modèle serait donc excellente mais son exactitude moyenne catastrophique. Les \gls{CNN} parviennent quant à eux à prédire correctement plusieurs types de véhicules, augmentant significativement l'exactitude moyenne tout en maintenant une exactitude globale compétitive. Des exemples qualitatifs de bonne segmentation mais mauvaise classification et de bonne segmentation et classification sont illustrés dans les~\cref{fig:negative_examples,fig:positive_examples}.

\begin{table}[t]
\centering
  \caption{Résultats de classification de véhicules sur les vérité terrain augmentées de Potsdam et Christchurch}
  \label{table:results_classif}
  \begin{tabular}{cccccccc}
  \toprule
  \textbf{Jeu de données} & \textbf{Classifieur} & \textbf{Voiture} & \textbf{Van} & \textbf{Camion} & \textbf{Pick-up} & \textbf{OA} & \textbf{AA}\\
  \midrule
  %\multirow{3}{*}{Potsdam} & AlexNet & 81\% & 42\% & \textbf{50\%} & \textbf{50\%} & 75\% & 56\%\\
  %& VGG & 83\% & 55\% & \textbf{50\%} & 33\% & 77\% & 55\%\\
  %& AlexNet+R & \textbf{99\%} & \textbf{66\%} & 25\% & 0\% & \textbf{83\%} & 48\%\\
  \multirow{3}{*}{Potsdam} & Cars only & 100\% & 0\% & 0\% & 0\% & 94\% & 25\%\\
  & AlexNet & \textbf{98\%} & \textbf{66\%} & 67\% & 0\% & \textbf{95\%} & 58\%\\
  & VGG-16 & 92\% & 66\% & \textbf{75\%} & \textbf{33\%} & 89\% & \textbf{67\%}\\
  \midrule
  \multirow{3}{*}{Christchurch} & Cars only & 100\% & 0\% & 0\% & 0\% & 94\% & 25\%\\
  & AlexNet & 94\% & 40\% & \textbf{67\%} & \textbf{89\%} & 93\% & 73\%\\
  & VGG-16 & \textbf{97\%} & \textbf{80\%} & \textbf{67\%} & 78\% & \textbf{96\%} & \textbf{80\%}\\
  \bottomrule
  \end{tabular}
\end{table}

L'exactitude moyenne est plus basse sur Potsdam que sur \gls{VEDAI} compte-tenu du fort déséquilibre entre classes et la sensibilité numérique des résultats. En effet, chaque sous-division entraînement/test de la validation croisée contient environ 15 exemples de camions et de pick-ups. Cependant, le modèle est entraîné sur \gls{VEDAI}, dont la répartition entre classes n'est pas autant déséquilibrée. Par conséquent, le modèle possède un biais qui ne se transfère pas à Potsdam. En outre, les capteurs utilisés pour \gls{VEDAI}, Potsdam et Christchurch sont différents, tout commes les environnements considérés (urbain pour Potsdam et Christchurch, rural pour \gls{VEDAI}).

Les variations dûes aux capteurs ont été corrigés en renormalisant la colorimétrie des images de Potsdam et Christchurch en utilisant les moments statistiques calculés sur \gls{VEDAI}\,:
\begin{equation}
X_{transformed} = \frac{X - m_{test~set}}{\sigma_{test~set}} \times \sigma_{train~set} + m_{train~set},
\end{equation}
avec $m$ la valeur du pixel moyen dans le jeu de données, $\sigma$ l'écart-type et $X$ l'image à transformer. Cette opération est appliquée sur chaque canal.

Toutefois, les écarts dûs à l'apparence de l'environnement et surtout aux types des véhicules diminuent tout de même les performances. Notamment, les voitures, camions et pick-ups sur Christchurch sont plus proches des marques américaines présentes dans \gls{VEDAI} que les véhicules de construction européenne présents dans Potsdam. Ces variations environnementales et d'apparence des objets peuvent faire sortir le classifieur de sa plage de fonctionnement.

Une meilleure régularisation ou un entraînement sur un jeu de données de véhicules plus varié pourrait permettre de limiter ces effets négatifs. De manière générale, ce type de transfert de connaissances est lié au problème d'adaptation de domaine non-supervisé~\cite{tuia_domain_2016,courty_optimal_2016}, qui est un sujet de recherche important en télédétection.

\subsection{Estimation de la densité de véhicules}

Une fois les véhicules extraits des images, une tâche simple à réaliser consiste à estimer leur nombre sur une zone donnée. Nous divisons les deux jeux de données en cellule de $1000\times1000$px (soit $125\times$\SI{125}{\meter\squared}) et nous comparons le nombre de véhicules prédits au nombre de véhicules contenus dans la vérité terrain\,:
\begin{equation}
\frac{|~\#~\text{predicted vehicles} - \#~\text{actual vehicles}~|}{\#~\text{actual vehicles}}.
\end{equation}

Les résultats sont moyennés et arrondis à l'entier le plus proche pour chaque jeu de données et détaillés dans le~\cref{table:results_counting}. Sur Potsdam comme Christchurch, les estimations sont justes avec moins de 10\% d'erreur ($\pm$5 véhicules en moyenne). Les estimations sur Christchurch ont une erreur légèrement supérieure compte-tenu de la segmentation et des détections moins précises.

\begin{table}[t]
\centering
  \caption{Erreur d'estimation moyenne des véhicules par cellule de 125$\times$\SI{125}{\meter\squared}.}
  \label{table:results_counting}
  \begin{tabular}{ccc}
  \toprule
  \textbf{Dataset} & \textbf{ISPRS Potsdam} & \textbf{NZAM/ONERA Christchurch}\\
  \midrule
  Erreur absolue (erreur moyenne/total dans la vérité terrain) & 3/52 & 6/66\\
  Erreur relative & 7.9\% & 9.1\%\\
  \bottomrule
  \end{tabular}
\end{table}

Une fois ce type de densité calculée, il est possible de réduire la taille des cellules et de produire des cartes de densité de présence de véhicules, comme illustrées dans les\cref{fig:vehicle_density_maps,fig:vehicle_density_maps_christchurch}. Ce type de cartes de densité peut ensuite être intégré à des \gls{SIG} comme OpenStreetMap pour identifier automatiquement des routes encombrées, des parkings~\cite{kamenetsky_aerial_2015}, etc.

\begin{figure}[t]
  \centering
  \begin{subfigure}{0.34\textwidth}
    \includegraphics[width=\textwidth]{images/potsdam_vehicle_3_rgb}
    \caption{}
  \end{subfigure}
  \hspace{0.1\textwidth}
  \begin{subfigure}{0.34\textwidth}
    \includegraphics[width=\textwidth]{images/potsdam_vehicle_3_pred_vehicles_dense}
    \caption{}
  \end{subfigure}
  \begin{subfigure}{0.34\textwidth}
    \includegraphics[width=\textwidth]{images/potsdam_vehicle_3_gt_vehicles}
    \caption{}
  \end{subfigure}
  \hspace{0.1\textwidth}
  \begin{subfigure}{0.34\textwidth}
    \includegraphics[width=\textwidth]{images/potsdam_vehicle_3_pred_vehicles}
    \caption{}
  \end{subfigure}\vspace{-12pt}

%   \begin{subfigure}{0.35\textwidth}
%     \includegraphics[angle=90,width=\textwidth]{images/vaihingen_vehicle_3_rgb}
%     \caption{RGB data (Vaihingen)}
%   \end{subfigure}
%   \hspace{0.1\textwidth}
%   \begin{subfigure}{0.35\textwidth}
%     \includegraphics[angle=90,width=\textwidth]{images/vaihingen_vehicle_3_pred_vehicles_dense}
%     \caption{Vehicle density map (Vaihingen)}
%   \end{subfigure}\\
%   \begin{subfigure}{0.35\textwidth}
%     \includegraphics[angle=90,width=\textwidth]{images/vaihingen_vehicle_3_gt_vehicles}
%     \caption{Vehicle ground truth (Vaihingen)}
%   \end{subfigure}
%   \hspace{0.1\textwidth}
%   \begin{subfigure}{0.35\textwidth}
%     \includegraphics[angle=90,width=\textwidth]{images/vaihingen_vehicle_3_pred_vehicles}
%     \caption{Predicted vehicles (Vaihingen)}
%   \end{subfigure}
  \caption{Visualisations de la répartition des véhicules sur Potsdam. (\textbf{a}) Image \glsname{RVB}; (\textbf{b}) densité de véhicules; (\textbf{c}) vérité terrain; (\textbf{d}) véhicules prédits.}
  \label{fig:vehicle_density_maps}
\end{figure}
\unskip
\begin{figure}[t]
  \centering
  \begin{subfigure}{0.35\textwidth}
    \includegraphics[width=\textwidth]{images/christchurch_vehicle_0_rgb}
    \caption{}
  \end{subfigure}
  \hspace{0.1\textwidth}
  \begin{subfigure}{0.35\textwidth}
    \includegraphics[width=\textwidth]{images/christchurch_vehicle_0_pred_vehicles_dense}
    \caption{}
  \end{subfigure}\\
  \begin{subfigure}{0.35\textwidth}
    \includegraphics[width=\textwidth]{images/christchurch_vehicle_0_gt_vehicles}
    \caption{}
  \end{subfigure}
  \hspace{0.1\textwidth}
  \begin{subfigure}{0.35\textwidth}
    \includegraphics[width=\textwidth]{images/christchurch_vehicle_0_pred_vehicles}
    \caption{}
  \end{subfigure}\vspace{-12pt}
  \caption{Visualisations de la répartition des véhicules sur Christchurch. (\textbf{a}) Image \glsname{RVB}; (\textbf{b}) densité de véhicules; (\textbf{c}) vérité terrain; (\textbf{d}) véhicules prédits.}
  \label{fig:vehicle_density_maps_christchurch}
\end{figure}

\section{Cartes de distances}

\subsection{Annotations sémantiques et cartes de distances}

La segmentation sémantique est une brique de base pour la compréhension de scène. En classifiant tous les pixels d'une image de façon dense, il est alors possible de construire des représentations abstraites s'intéressant aux objets et à leurs formes.
%Semantic segmentation is a task that is of paramount importance for visual scene understanding. It is often used as the first layer to obtain representations of a scene with a high level of abstraction, such as listing objects and their shapes.

Les réseaux entièrement convolutifs (\emph{Fully Convolutional Networks} ou FCN) sont un outil particulièrement efficace pour la segmentation sémantique pour de nombreux types d'images\,: multimédia~\cite{everingham_pascal_2014}, aériennes~\cite{rottensteiner_isprs_2012}, médicales~\cite{ulman_objective_2017} ou de véhicule autonome~\cite{cordts_cityscapes_2016}.
%Fully Convolutional Networks (FCN) have proved themselves to be very effective for semantic segmentation of all kinds of images, from multimedia images~\cite{everingham_pascal_2014} to remote sensing data~\cite{rottensteiner_isprs_2012}, medical imaging~\cite{ulman_objective_2017} and autonomous driving~\cite{cordts_cityscapes_2016}.

Cependant, la littérature fait régulièrement face à des problèmes de frontières inter-classes imprécises ou de segmentations bruitées, nécessitant de faire appel à des régularisations a posteriori pour lisser les segmentations~\cite{zheng_conditional_2015,l._c._chen_deeplab:_2017}.
%However, a recurrent issue often raised by the practitioners is the fact that FCN tend to produce blurry or noisy segmentations, in which spatial transitions between classes are not as sharp as expected and objects sometimes lack connectivity or convexity, and therefore the results need to be regularized using some post-processing~\cite{zheng_conditional_2015,l._c._chen_semantic_2016}.
La communauté s'est ainsi penchée sur différents post-traitements pour améliorer la netteté des contours et contraindre les segmentations à respecter la même topologie que la vérité terrain. Bien souvent, il s'agit de modèles graphiques ajoutés en fin de réseau~\cite{z._liu_deep_2017} ou faisant appel à des connaissances a priori~\cite{le_reformulating_2017,bertasius_semantic_2016}.
%This has led the computer vision community to investigate many post-processing and regularization techniques to sharpen the visual boundaries and enforce spatial smoothness in semantic maps inferred by FCNs. Yet these methods are often either graphical models added on top of deep neural networks~\cite{z._liu_deep_2017,zheng_conditional_2015} or based on sophisticated prior knowledge~\cite{le_reformulating_2017,bertasius_semantic_2015}.

Ce travail propose une approche directe consistant en une régularisation implicite intégrée dans la représentation de la vérité terrain. En effet, nous proposons d'utiliser les cartes de distance issues des masques de segmentation comme tâche auxiliaire. Les cartes de distance indiquent non seulement l'appartenance d'un pixel à une classe donnée, mais également sa proximité spatiale vis-à-vis des autres classes d'intérêt et contient donc une information plus riche concernant la structure spatiale des données. Cette approche s'inscrit dans la veine de travaux sur l'utilisation de primitives géométriques pour régulariser la segmentation sémantique, comme la prédiction de l'orientation des objets~\cite{uhrig_pixel-level_2016} ou de la position de leur centre de masse~\cite{hayder_boundary-aware_2017}
%In this work, we propose a much straightforward approach by introducing a simple implicit regularization embedded in the network loss function. We consider the distance transform of the segmentation masks in a regression problem as a proxy for the semantic segmentation task.
%The distance transform is a continuous representation of the label masks, where one pixel becomes represented not only by its belonging to a class, but by its spatial proximity to all classes. This means that the gradient back-propagated contains more information about the underlying spatial structure of the data compared to traditional classification.
De fait, en modifiant de façon minimale des réseaux de segmentation existants, nous parvenons à obtenir des segmentation plus régulières sans post-traitement ou connaissance a priori.
%As such, the network learns a smoother segmentation with a very low complexity overhead.
%Moreover, this is straightforward to implement and does not rely on any additional priors, but only on an alternative representation of the ground truth. Therefore any deep segmentation architecture can be adapted in this fashion without any structural alteration.
Nous validons notre méthode sur plusieurs architectures de réseaux convolutifs profonds et sur plusieurs applications en compréhension de scène urbaine, en segmentation d'images 2,5D et en observation de la Terre.
%We validate our method with several architectures on diverse application domains on which we obtain significant improvements w.r.t strong baselines: urban scene understanding, RGB-D images and Earth Observation.

\subsection{Régularisation par régression des cartes de distances}
\begin{figure}[!t]
    \begin{subfigure}{0.33\textwidth}
    	\includegraphics[width=\textwidth]{maskcars_colorbar}
        \caption{Masque binaire de segmentation.}
    \end{subfigure}
    \begin{subfigure}{0.33\textwidth}
    	\includegraphics[width=\textwidth]{distcars_colorbar}
        \caption{Carte de distance signée (CDS).}
    \end{subfigure}
    \begin{subfigure}{0.33\textwidth}
    	\includegraphics[width=\textwidth]{distcars_normalized_colorbar}
        \caption{CDS tronquée et normalisée.}
    \end{subfigure}

	\begin{subfigure}{0.33\textwidth}
    	\includegraphics[width=\textwidth]{potsdam_maskroad_zoom}
        \caption{Masque binaire de segmentation.}
    \end{subfigure}
    \begin{subfigure}{0.33\textwidth}
    	\includegraphics[width=\textwidth]{potsdam_distroad_zoom_colorbar}
        \caption{Carte de distance signée (CDS).}
    \end{subfigure}
    \begin{subfigure}{0.33\textwidth}
    	\includegraphics[width=\textwidth]{potsdam_distroadnorm_zoom_colorbar}
        \caption{CDS tronquée et normalisée.}
    \end{subfigure}
    \caption{Différentes représentations de segments annotés.}
    \label{fig:representations}
\end{figure}

Ce travail s'intéresse à l'utilisation des cartes de distances signées (CDS) pour régulariser des réseaux de segmentation sémantique. Le passage en carte de distances transforme un masque binaire en une représentation équivalente mais à valeurs continues. En l'occurence, nous travaillons avec des cartes de distances signées tronquées puis renormalisées dans $[-1,1]$. Ces représentations des annotations sont illustrées dans la~\cref{fig:representations}. Nous émettons l'hypothèse que cette représentation permet toutefois d'accéder plus directement à la structure spatiale des données, notamment car elle contient pour un pixel sa distance spatiale relative à toutes les classes d'intérêt. Cette représentation est donc plus riche en quantité d'information que les masques binaires utilisés pour la classification. Nous montrons que l'utilisation de la régression des CDS en tâche auxiliaire d'un réseau de segmentation sémantique a des résultats bénéfiques sur la segmentation finale.
%In this work, we suggest to use the signed distance transform (SDT) to improve the learning process of semantic segmentation models. The SDT transforms binary sparse masks into equivalent continuous representations. We argue that this representation is more informative for training deep networks as one pixel now owns a more precise representation of its spatial proximity with various semantic classes. We show that using a multi-task learning framework, we can train FCN to perform both semantic segmentation by traditional classification and SDT regression, and this helps the network infer better structured semantic maps.

%\subsection{Signed-distance transform}
% The distance transform of a binary image assigns to every pixel in the grid its distance to the nearest foreground point. The distance can be computed using several metrics, such as the Manhattan distance or the Euclidean distance. For foreground elements, the distance is set to 0. For example, the Euclidean transform of an image $X$ with a foreground delimited by the mask $M$ will be $D$ such as:\sl{revoir l'equation / notations}
% \begin{equation}
% \forall i,j,~~~d_{i,j} = min_{z \in M}(\parallel x_{i,j} - z \parallel).
% \end{equation}

%We use the signed distance transform (SDT)~\cite{q._z._ye_signed_1988}, which assigns to each pixel of the foreground its distance to the closest background point, and to each pixel of the background the opposite of its distance to the closest foreground point. If $x_{i,j}$ are the input image pixel values and $M$ the foreground mask, then the pixels $d_{i,j}$ of the distance map are obtained with the following equation:
%\begin{equation}
%\forall i,j,~~~d_{i,j} =
%\begin{cases}
%+ min_{z \notin M}(\parallel x_{i,j} - z \parallel), & \text{if}~x_{i,j} \in M,\\
%- min_{z \in M}(\parallel x_{i,j} - z \parallel), & \text{if}~x_{i,j} \notin M.\\
%\end{cases}
%\end{equation}

%Considering that semantic segmentation annotations can be interpreted as binary masks, with one mask per class, it is possible to convert the labels into their signed-distance transform counterparts. In this work, we apply class-wise the signed Euclidean distance transform to the labels using a linear time exact algorithm~\cite{c._r._maurer_linear_2003}.

%In order to avoid issues where the nearest point is outside the receptive field of the network, we clip the distance to avoid long-range spatial dependencies that would go out of the network field-of-view. We then normalize the SDT to constrain it in the $[-1;1]$ range. This can be seen as feeding the SDT into a non-linear saturating activation function $hardtanh$. The visual representations are illustrated in~\cref{fig:representations}. The same processing is applied to the distances estimated by the network.

\begin{figure*}[!t]
	\centering
	\includegraphics[width=0.9\textwidth]{multitask_distances}
    \caption{Apprentissage multi-tâches (classification pixel à pixel et régression des cartes de distances). Les couches convolutives sont en {\color{blue}bleu}, les activations non-linéaires en {\color{green!50!black}vert} et les cartes d'activation en {\color{brown}marron}.}
    \label{fig:distance_framework}
\end{figure*}

%\subsection{Multi-task learning}

%Signed distance transform maps are continuous representations of the labels. We can train a deep network to approximate these maps using a regression loss.

La régression directe des CDS ne permet pas d'obtenir de meilleurs résultats de segmentation que la classification dense pixel à pixel usuelle. De fait, nous proposons donc d'utiliser une stratégie d'apprentissage multi-tâches dans laquelle le réseau est optimisé à la fois sur la classification des pixels et sur la régression des CDS.
%However, preliminary experiments show that training only for regression does not bring any improvement compared to traditional classification and even degrades the results.
%sl% penses-tu qu'il faudrait donner des résultats ici ou juste le dire?
%Therefore, we suggest to use a multi-task strategy, in which the network learns both the classification on the usual one-hot labels and the regression on the SDT.

En particulier, nous modifions l'architecture du réseau pour, dans un premier temps, effectuer la régression des CDS; puis nous ajoutons une couche convolutive additionnelle pour fusionner les activations de la dernière couche avec les CDS prédites afin de réaliser la classification finale. Le réseau est ainsi entraîné en multi-tâches, la régression des CDS étant utilisée comme tâche intermédiaire avant la classification.
%More precisely, we alter the network to first predict the SDT and we then use an additional convolutional layer to fuse the last layer features and the inferred SDT to perform the final classification. In this way, the network is trained in a cascaded multi-task fashion, where the distance transform regression is used as a proxy, i.e. an intermediate task, before classification.

L'altération du réseau se résume comme suit. La dernière couche, habituellement suivie d'un \textit{softmax} est ici utilisée comme couche de régression des CDS. Les distances étant normalisées entre $-1$ et $1$, la fonction $hardtanh$ est utilisée comme activation non-linéaire. Puis nous concaténons les activations de la couche précédente aux CDS ainsi prédites pour alimenter une couche convolutive additionnelle suivie d'un \textit{softmax}. L'architecture complète est illustrée dans la~\cref{fig:distance_framework}.
%Therefore, the network modification can be summarized as follows. Instead of using the last layer and feeding it into a softmax, we now use the last layer as a distance prediction. As distances are normalized between -1 and 1, these distances pass through a $hardtanh$ non-linearity. Then, we concatenate the previous layer features maps and the distance predictions to feed both into a convolutional and a softmax layer. The complete architecture is illustrated in~\cref{fig:distance_framework}.

Par souci d'équité dans nos expériences, les modèles de référence présentés dans la~\cref{sec:experiments} utilisent également une couche de convolution additionnelle afin que tous les modèles équivalents possèdent le même nombre de paramètres optimisables.
%For a fair comparison, the classification baselines considered in~\cref{sec:experiments} use the same additional convolutional layer, so that both models have the same number of parameters.

Les fonctions de coût utilisées dans ces travaux sont la log-vraisemblance négative (NLL) pour la classification et la distance L1 pour la régression.
%In this work, we keep the traditional cross-entropy loss for classification, in the form of the negative log-likelihood (NLL). As our regression results are constrained in $[-1;1]$, we use the L1 loss to preserve relative errors.

En notant respectivement $Z_{seg}, Z_{dist}, Y_{seg}, Y_{dist}$ la classification après \textit{softmax}, la carte de distances prédite, les annotations de vérité terrain et la carte de distances réelle, la fonction de coût à minimiser est\,:
%Assuming that $Z_{seg}, Z_{dist}, Y_{seg}, Y_{dist}$ respectively denote the output of the segmentation softmax, the regressed distance, the ground truth segmentation labels and the ground truth distances, the final loss to be minimized is:
\begin{equation}
L = NLL(Z_{seg}, Y_{seg}) + \lambda L1(Z_{dist}, Y_{dist})
\end{equation}
%where $\lambda$ is an hyper-parameter that controls the strength of the regularization.
où $\lambda$ est un hyperparamètre contrôlant l'amplitude de la régularisation.

\subsection{Références}

Afin de pouvoir mesurer l'effet de la régression des cartes de distance, nous entraînons des réseaux avec l'architecture SegNet~\cite{badrinarayanan_segnet:_2017} ou PSPNet~\cite{zhao_pyramid_2017} de référence, soit en régression pure, soit en classification pure.
%We first obtain baseline results on various datasets using SegNet or PSPNet for semantic segmentation, either using the cross entropy for label classification or the L1 loss for distance regression. However, note that our method is not architecture-dependent. It consists in a straightforward modification of the end of the network that would fit any architecture designed for semantic segmentation.

SegNet~\cite{badrinarayanan_segnet:_2017} est un modèle encodeur-décodeur conçu pour la conduite autonome. Sa conception dérive du modèle VGG-16~\cite{simonyan_very_2014}. L'encodeur produit des cartes d'activation à résolution $1{:}32$. Ces cartes sont ensuite suréchantillonnées et projetées dans l'espace sémantique par le décodeur.
%SegNet~\cite{badrinarayanan_segnet:_2017} is a popular architecture for semantic segmentation, originally designed for autonomous driving. It is designed around a symmetrical encoder-decoder architecture based on VGG-16~\cite{simonyan_very_2014}. The encoder results in downsampled feature maps at 1:32 resolution. These maps are then upsampled and projected in the label space by the decoder using unpooling layer. The unpooling operation replaces the decoded activations into the positions of the local maxima computed in the encoder maxpooling layers.

PSPNet~\cite{zhao_pyramid_2017} est une architecture de segmentation sémantique récente ayant établi un nouvel état de l'art sur plusieurs jeux de données~\cite{cordts_cityscapes_2016,everingham_pascal_2014}. Elle dérive du modèle ResNet~\cite{he_deep_2016} et utilise un module de concaténation en pyramide d'activations pour prendre en compte plusieurs niveaux de contexte spatial. Dans notre cas, nous utilisons une version réduite de PSPNet conçue sur l'architecture ResNet-101. ResNet-101 produit des cartes d'activation à résolution $1{:}32$ qui sont suréchantillonnées par déconvolution.
%PSPNet~\cite{zhao_pyramid_2017} is a recent model for semantic segmentation that achieved new state-of-the-art results on several datasets~\cite{cordts_cityscapes_2016,everingham_pascal_2014}. It is based on the popular ResNet~\cite{he_deep_2016} model and uses a pyramidal module at the end to incorporate multi-scale contextual cues in the learning process. In our case, we use a smaller albeit efficient version of PSPNet based on ResNet-50~\cite{he_deep_2016}. ResNet-50 encodes the input into feature maps at 1:32 resolution, which are then upsampled using transposed convolutions.

\subsection{Jeux de données}

\begin{table*}[!t]
\begin{tabularx}{\textwidth}{c Y Y Y Y Y Y Y}
\toprule
Méthode & Ville & \% classification & Routes & Bâtiments & Vég. basse & Arbres & Véhicules\\
\midrule
% Regression (lambda = +inf) OA : 89.49 +- 0.86 F1 : 91.03 +- 1.35 F1 : 95.60 +- 0.26 F1 : 81.23 +- 2.76 F1 : 88.31 +- 1.99 F1 : 0.00 +- 0.00
SegNet* (régression) & \multirow{3}{*}{Vaihingen} & 89.49 & 91.03 & 95.60 & 81.23 & 88.31 & 0.00\\
% Classif (lambda = 0) : OA : 90.00 +- 1.17 F1 : 91.98 +- 1.62 F1 : 95.53 +- 0.20 F1 : 80.91 +- 2.62 F1 : 88.07 +- 2.33 F1 : 87.94 +- 1.76
SegNet* (classification) & & 90.00 & 91.98 & 95.53 & 80.91 & 88.07 & 87.94\\
% Multi-task (lambda = 2) : OA : 90.43 +- 1.00 F1 : 92.46 +- 1.41 F1 : 95.99 +- 0.24 F1 : 81.30 +- 2.96 F1 : 88.34 +- 2.09 F1 : 88.16 +- 1.24
SegNet* (multi-tâches) & & \textbf{90.43} & \textbf{92.46} & \textbf{95.99} & \textbf{81.30} & \textbf{88.34} & \textbf{88.16}\\
\midrule
%SegNet (regression) & \multirow{3}{*}{Potsdam} & ? & ? & ? & ? & ? & ?\\
% Classif (lambda = 0) OA : 91.85 +- 1.19 F1 : 94.12 +- 0.37 F1 : 96.09 +- 1.02 F1 : 88.48 +- 0.56 F1 : 85.44 +- 1.93 F1 : 96.62 +- 0.30
SegNet* (classification) & \multirow{2}{*}{Potsdam} & 91.85 & 94.12 & 96.09 & 88.48 & 85.44 & 96.62\\
% Multi-task (lambda = 2) : OA : 92.22 +- 1.15 F1 : 94.33 +- 0.45 F1 : 96.52 +- 1.03 F1 : 88.55 +- 0.62 F1 : 86.55 +- 1.28 F1 : 96.79 +- 0.37
SegNet* (multi-tâches) & & \textbf{92.22} & \textbf{94.33} & \textbf{96.52} & \textbf{88.55} & \textbf{86.55} & \textbf{96.79}\\
\bottomrule
\end{tabularx}
\caption{Résultats de validation croisée sur les jeux de données ISPRS. Les valeurs indiquées représentent le taux global de bonne classification et le score F1 pour chaque classe.}
\label{tab:isprs_results}
\end{table*}

Nous validons notre approche sur plusieurs jeux de données afin de démontrer sa capacité à généraliser dans des contextes de segmentation mono et multi-classes sur plusieurs types d'images.
%We validate our method on several datasets in order to show its generalization capacity on multi and mono-class segmentation of both ground and aerial images.

\paragraph{ISPRS 2D Semantic Labeling}

Le jeu de données ISPRS 2D Semantic Labeling~\cite{rottensteiner_isprs_2012} est constitué de deux ensembles d'images aériennes. Le premier a été acquis sur la ville de Vaihingen et comporte 33 images infrarouge-rouge-vert (IRRV) à une résolution de 9cm/px d'une taille moyenne d'environ $2000\times1500$px. Des annotations denses sur 16 images sont disponibles sur les six classes suivantes\,: routes, bâtiments, végétation basse, arbres, véhicules ainsi qu'une classe ``autre''.
%The ISPRS 2D Semantic Labeling~\cite{rottensteiner_isprs_2012} datasets consist in two sets of aerial images.
%The Vaihingen scene is comprised of 33 infrared-red-green (IRRG) tiles with a spatial resolution of 9cm/px, with an average size of $2000\times1500$px. Dense annotations are available on 16 tiles for six classes: impervious surfaces, buildings, low vegetation, trees, cars and clutter, although the latter is not included in the evaluation process.
Le second ensemble a été acquis sur la ville de Potsdam et comporte 38 images infrarouge-rouge-vert-bleu (IRRVB) à une résolution de 5cm/px, pour une taille moyenne de $6000\times6000$px. Des annotations denses sont disponibles pour les mêmes classes que précédemment sur 24 images.
%The Potsdam scene is comprised of 38 infrared-red-green-blue (IRRGB) tiles with a spatial resolution of 5cm/px and size $6000\times6000$px. Dense annotations for the same classes are available on 24 tiles.
L'évaluation est réalisée par validation croisée en divisant les jeux de données en trois subdivisions.
%Evaluation is done by splitting the datasets with a 3-fold cross-validation.
%Training samples are $256\times256$ patches randomly extracted from the training tiles with horizontal and vertical symmetries. Inference is done by sliding a $256\times256$ window with a stride of 32px on the test tiles.

\paragraph{INRIA Aerial Image Labeling Benchmark}
Le jeu de données INRIA Aerial Image Labeling~\cite{maggiori_can_2017} contient 360 images RVB de taille $5000\times5000$px à une résolution de 30cm/px, couvrant 10 agglomérations de divers points du globe. La moitié des villes sont utilisées pour l'apprentissage et associées à des annotations publiques d'empreintes de bâtiments. Le reste du jeu de données est utilisé pour l'évaluation.
%The INRIA Aerial Image Labeling dataset~\cite{maggiori_can_2017} is comprised of 360 RGB tiles of $5000\times5000$px with a spatial resolution of 30cm/px on 10 cities across the globe. Half of the cities are used for training and are associated to a public ground truth of building footprints. The rest of the dataset is used only for evaluation with a hidden ground truth.
%Training samples are $384\times384$ patches randomly extracted from the training tiles with horizontal and vertical symmetries. Inference is done by sliding a $384\times384$ window with a stride of 96px on the test tiles.

\paragraph{SUN RGB-D}
Le jeu de données SUN RGB-D~\cite{song_sun_2015} contient 10 335 images RVB accompagnées d'une carte de profondeur. Ces images ont été annotées sur 37 classes d'intérêt comportant le mobilier, les murs, le sol\dots
%The SUN RGB-D dataset~\cite{song_sun_2015} is comprised of 10,335 RGB-D images of indoor scenes acquired from various sensors, each capturing a color image and a depth map. These images have been annotated for 37 semantic classes such as ``chairs'', ``floor'', ``wall'' or ``table'', with a few pixels unlabeled.% The input images are resized to $224\times224$px and the predictions are upsampled at inference time.

%\paragraph{Data Fusion Contest 2015}
%The Data Fusion Contest 2015~\cite{campos-taberner_processing_2016} is comprised of 7 aerial RGB images of $10,\!000\times10,\!000$px with a spatial resolution of 5cm/px on the city of Zeebruges, Belgium. A dense set of annotations on 8 classes (6 from ISPRS dataset plus ``water'' and ``boat'') is given. Two images are reserved for testing, we use one image for validation and the rest for training.
%We extract $256\times256$ patches from the training tiles, augmented by vertical and horizontal symmetries. Inference is done by sliding a $256\times256$ window with a stride of 32px on the test tiles.

\paragraph{CamVid}
Le jeu de données CamVid~\cite{brostow_semantic_2009} comporte 701 images extraites de plusieurs vidéos filmées par une caméra embarquée dans une voiture, avec une résolution de $360\times480$px. Nous utilisons la même division du jeu de données que~\cite{badrinarayanan_segnet:_2017}, c'est-à-dire 367 images d'apprentissage, 101 images de validation et 233 images de test. Les annotations recouvrent 11 classes d'intérêt telles que ``bâtiment'', ``piéton'', ``voiture'' ou encore ``trottoir''.
%The CamVid dataset~\cite{brostow_semantic_2009} is comprised of 701 fully annotated still frames from urban driving videos, with a resolution of $360\times480$px. We use the same split as in~\cite{badrinarayanan_segnet:_2017}, \textit{i.e.} 367 training images, 101 validation images and 233 test images. The ground truth covers 11 classes relevant to urban scene labeling, such as ``building'', ``road'', ``car'', ``pedestrian'' and ``sidewalk''. A few pixels are assigned to a void class that is not evaluated.

%\subsubsection{Cityscapes}
%The Cityscapes dataset~\cite{cordts_cityscapes_2016} is focused on autonomous driving and is comprised of 5,000 images taken from a moving car in several German cities, with a resolution of $2048\times1024$. Dense annotations are available for 3,000 images in the training set and 500 images in the validation set, while the remaining 1,500 are used as a held-out test set. The optional coarse annotations on an additional set of 20,000 images are not used in this work. Models are evaluated on 18 classes. Although the labels are dense, some pixels are not evaluated, e.g. the ego-car and some clutter objects.
%The training images are resized to $512\times256$. Inference is done in the same fashion and the results are upsampled to the full image resolution.

\subsection{Protocole expérimental}

Nous expérimentons avec les modèles SegNet et PSPNet-101.
%We experiment with the SegNet and PSPNet models.

SegNet est entraîné pendant 50\,000 itérations sur des \emph{mini-batches} de 10 images. L'optimisation se fait par descente de gradient stochastique avec un taux d'apprentissage de 0,01, divisé par 10 après 25\,000 et 45\,000 itérations. Les poids de l'encodeur sont initialisés avec ceux de VGG-16~\cite{simonyan_very_2014} pré-entraîné sur ImageNet. Les poids du décodeur sont initialisés aléatoirement en utilisant la stratégie proposée dans~\cite{he_delving_2015}.
%SegNet is trained for 50 epochs with a batch size of 10. Optimization is done using Stochastic Gradient Descent (SGD) with a base learning rate of 0.01, divided by 10 after 25 and 45 epochs, and a weight decay set at 0.0005. Encoder weights are initialized from VGG-16~\cite{simonyan_very_2014} trained on ImageNet~\cite{deng_imagenet:_2009}, while decoder weights are initialized using the policy from~\cite{he_delving_2015}.
Pour le jeu de données multi-modal SUN RGB-D, nous utilisons le modèle FuseNet~\cite{hazirbas_fusenet:_2016}, qui consiste en un SegNet à double entrée.
%For SUN RGB-D, in order to validate our method in a multi-modal setting, we use the FuseNet~\cite{hazirbas_fusenet:_2016} architecture. This model consists in a dual-stream SegNet that learns a joint representation of both the color image and the depth map. We train it using SGD with a learning rate of 0.01 on resized $224\times224$ images.
Sur les images aériennes, nous augmentons le nombre d'échantillons d'apprentissage en extrayant des images de $256\times256$ ($384\times384$ pour le jeu de données INRIA Aerial Image) et en procédant aléatoirement à des symétries horizontales ou verticales. L'inférence est réalisée avec une fenêtre glissante de même dimension et un recouvrement de 75\%.
%On aerial images, we randomly extract $256\times256$ ($384\times384$ on the INRIA Labeling dataset), augmented with flipping and mirroring. Inference is done using a sliding window of the same shape with a 75\% overlap.

PSPNet est entraîné sur CamVid pendant 750\,000 itérations sur 10 images en parallèle par descente de gradient stochastique avec un taux d'apprentissage de 0,01, divisé par 10 après 500\,000 itérations. Nous extrayons aléatoirement des imagettes de $224\times224$ et nous appliquons aléatoirement une symétrie horizontale. Suivant le protocole de~\cite{jegou_one_2017}, nous raffinons l'apprentissage en entraînant pendant 200\,000 itérations sur les images à pleine résolution.
%We train a PSP-Net on CamVid for 750 epochs using SGD with a learning rate of 0.01, divided by 10 at epoch 500, a batch size of 10 and a weight decay set at 0.0005. We extract random $224\times224$ crops from the original images and we perform random mirroring to augment the data. We fine-tune on full scale images for 200 epochs, following the practice from~\cite{jegou_one_2017}.
Notre implémentation de PSPNet utilise les poids de ResNet-101~\cite{he_deep_2016} pré-entraînés sur ImageNet pour l'initialisation, et n'utilise pas la fonction de coût auxiliaire présentée dans~\cite{zhao_pyramid_2017}.
%Our implementation of PSPNet is based on ResNet-50 pre-trained on ImageNet and do not use the auxiliary classification loss for deep supervision~\cite{zhao_pyramid_2017}.

Finalement, nous compensons le déséquilibre des classes dans les jeux de données SUN RGB-D et CamVid en utilisant une pondération relativement à la fréquence médiane.
%Finally, we use median-frequency balancing to alleviate the class unbalance from SUN RGB-D and CamVid.

Toutes les expériences sont réalisées à l'aide de la bibliothèque PyTorch~\cite{noauthor_pytorch:_2016}. Les CDS sont calculées sur CPU à l'aide de la bibliothèque Scipy~\cite{jones_scipy:_2001} et conservées en mémoire pour éviter les calculs inutiles.
%All experiments are implemented using the PyTorch library~\cite{noauthor_pytorch:_2016}. SDT is computed on CPU using the Scipy library~\cite{jones_scipy:_2001} and cached on-memory or on-disk, which slows down training during the first epoch and use system resources. Online SDT computation using a fast GPU implementation~\cite{zampirolli_fast_2017} would strongly alleviate those drawbacks.

\begin{figure*}[t]
\begin{subfigure}{0.24\textwidth}
	\includegraphics[width=\textwidth]{images_vaihingen/top_mosaic_09cm_area32}
    \caption{Image IRRV}
\end{subfigure}
\begin{subfigure}{0.24\textwidth}
	\includegraphics[width=\textwidth]{images_vaihingen/top_mosaic_09cm_area32_gt_colors}
    \caption{Vérité terrain}
\end{subfigure}
\begin{subfigure}{0.24\textwidth}
	\includegraphics[width=\textwidth]{images_vaihingen/result_segnet_256_proxy_0_32_colors}
    \caption{SegNet (classification)}
\end{subfigure}
\begin{subfigure}{0.24\textwidth}
	\includegraphics[width=\textwidth]{images_vaihingen/result_segnet_256_proxy_2_32_colors}
    \caption{SegNet (multi-tâches)}
\end{subfigure}
\caption{Extrait des résultats de segmentation sur le jeu de données ISPRS Vaihingen. \small{Légende\,: blanc\,: routes, \textcolor{Blue}{bleu}\,: bâtiments, \textcolor{Cerulean}{cyan}: végétation basse, \textcolor{OliveGreen}{vert}\,: arbres, \textcolor{Dandelion}{jaune}: véhicules, \textcolor{BrickRed}{rouge}: autre}, noir\,: indéfini.}
\label{fig:isprs_vaihingen}
\end{figure*}

\begin{figure*}[t]
\begin{subfigure}{0.24\textwidth}
	\includegraphics[width=\textwidth]{images_potsdam/top_potsdam_2_11_RGB}
    \caption{Image RVB}
\end{subfigure}
\hfill
\begin{subfigure}{0.24\textwidth}
	\includegraphics[width=\textwidth]{images_potsdam/top_potsdam_2_11_label_gt_colors}
    \caption{Vérité terrain}
\end{subfigure}
\hfill
\begin{subfigure}{0.24\textwidth}
	\includegraphics[width=\textwidth]{images_potsdam/result_segnet_potsdam_256_proxy_0_2_11_colors}
    \caption{SegNet (classification)}
\end{subfigure}
\hfill
\begin{subfigure}{0.24\textwidth}
	\includegraphics[width=\textwidth]{images_potsdam/result_segnet_potsdam_256_proxy_2_2_11_colors}
    \caption{SegNet (multi-tâches)}
\end{subfigure}
\caption{Extrait des résultats de segmentation sur le jeu de données ISPRS Potsdam. \small{Légende\,: blanc\,: routes, \textcolor{Blue}{bleu}\,: bâtiments, \textcolor{Cerulean}{cyan}: végétation basse, \textcolor{OliveGreen}{vert}\,: arbres, \textcolor{Dandelion}{jaune}: véhicules, \textcolor{BrickRed}{rouge}: autre}, noir\,: indéfini.}
\label{fig:isprs_potsdam}
\end{figure*}

\subsection{Résultats}

Dans les~\cref{tab:isprs_results,tab:inria_results,tab:sun_results,tab:camvid_results}, les modèles suffixés par ``*'' sont ceux proposés dans le cadre cette étude.

\paragraph{ISPRS dataset}
Les résultats de validation croisée sur les jeux de données ISPRS Vaihingen et Potsdam sont détaillés dans le~\cref{tab:isprs_results}. Toutes les classes semblent bénéficier de la régression des cartes de distances. En particulier, les arbres sur Potsdam sont significativement mieux segmentés, la régression de la CDS contraignant le réseau à prendre en compte la convexité naturelle de l'objet en dépit de l'absence de feuilles. Deux exemples de segmentation sont présentés en~\cref{fig:isprs_vaihingen} et~\cref{fig:isprs_potsdam}, dans lesquelles on peut voir que les bâtiments bénéficient grandement de l'approche multi-tâches (apparence plus lisse et moins de bruit de classification). Nous avons également testé l'approche par régression seule sur le jeu de données ISPRS Vaihingen avec des résultats mitigés. En effet, la plupart des classes bénéficient de ce traitement mais le réseau devient alors incapable de segmenter les véhicules, provoquant dans l'ensemble une baisse du taux de bonne classification.
%The cross-validated results on the ISPRS Vaihingen and Potsdam datasets are reported in~\cref{tab:isprs_results}. All classes seem to benefit from the distance transform regression. On Potsdam, the class ``trees'' is significantly improved as the distance transform regression forces the network to better learn its closed shape, despite the absence of leaves that make the underlying ground visible from the air. Two example tiles are shown in~\cref{fig:isprs_vaihingen} and~\cref{fig:isprs_potsdam}, where most buildings strongly benefit from the distance transform regression, with smoother shapes and less classification noise. Moreover, we also tested to perform regression only on the Vaihingen dataset, which slightly improved the results on several classes, although it missed all the cars and had a negative impact overall.
%It is also worth noting that our strategy succeeds while CRF did not improve classification results on this dataset as reported in~\cite{marmanis_classification_2017}.

\paragraph{INRIA Aerial Image Labeling Benchmark}

\begin{table}
\begin{tabularx}{0.49\textwidth}{c Y c}
\toprule
Méthode & I/U & \% classification\\
\midrule
%SegNet~\cite{bischke_multi-task_2017} & 72.57 & 95.66\\
%SegNet (multi-task)~\cite{bischke_multi-task_2017} & 73.00 & 95.73\\
SegNet* (classification) & 65.04 & 94.74\\
SegNet* (multi-tâches) & \textbf{71.02} & \textbf{95.63}\\
\bottomrule
\end{tabularx}
\caption{Résultats sur le jeu de données INRIA Aerial Image Labeling. Nous indiquons le taux global de bonne classification ainsi que le ratio intersection sur union (I/U).}
\label{tab:inria_results}
\end{table}

\begin{figure*}[t]
\begin{subfigure}{0.24\textwidth}
	\includegraphics[width=\textwidth]{images_inria/chicago5_rgb_zoom}
    \caption{RGB image}
\end{subfigure}
\begin{subfigure}{0.24\textwidth}
	\includegraphics[width=\textwidth]{images_inria/chicago5_gt_zoom}
    \caption{Ground truth}
\end{subfigure}
\begin{subfigure}{0.24\textwidth}
	\includegraphics[width=\textwidth]{images_inria/chicago5_standard_zoom}
    \caption{SegNet (classification)}
\end{subfigure}
\begin{subfigure}{0.24\textwidth}
	\includegraphics[width=\textwidth]{images_inria/chicago5_proxy_zoom}
    \caption{SegNet (multi-task)}
\end{subfigure}
\caption{Extrait des résultats de segmentation sur le jeu de données INRIA Aerial Image Labeling. Les pixels corrects sont en \textcolor{OliveGreen}{vert}, les faux positifs en \textcolor{Lavender}{rose} et les faux négatifs en \textcolor{Blue}{bleu}. L'approche multi-tâches capture mieux la structure spatiale des objets.}
\label{fig:inria_results}
\end{figure*}

Les résultats sur le jeu de données INRIA Aerial Image Labeling sont détaillés dans le~\cref{tab:inria_results}. L'utilisation de la régression sur les cartes de distances améliore significativement le ratio d'intersection sur union. Comme illustré dans la~\cref{fig:inria_results}, les formes des bâtiments respectent mieux l'a priori polygonal et la connexité des objets. Les bâtiments qui étaient déjà détectés sont segmentés avec plus de régularité.
%The results on the validation set of the INRIA Aerial Image Labeling benchmark are reported in~\cref{tab:inria_results}. Using the distance transform regression improves the intersection over union (IoU) by 0.47 and makes many errors disappear. As shown in~\cref{fig:inria_results}, the buildings shapes are more regular in the multi-task prediction and fit better to the original image edges. Although no additional buildings are detected, those that were already segmented become cleaner. Compared to~\cite{bischke_multi-task_2017}, we have a stronger baseline that is even more improved by the distance transform regression using a simpler method than the proposed quantized distance classification.

\paragraph{SUN RGB-D}
\begin{table}
\setlength{\tabcolsep}{3pt}
\begin{tabularx}{0.48\textwidth}{Y c c c}
\toprule
Méthode & \% classification & I/U & Précision\\
\midrule
3D Graph CNN~\cite{qi_3d_2017} & - & 42.0 & 55.2\\
3D Graph CNN~\cite{qi_3d_2017} (multi-échelles) & - & \textbf{43.1} & 55.7\\
\midrule
FuseNet*~\cite{hazirbas_fusenet:_2016} & 76.8 & 39.0 & 55.3\\
FuseNet* (multi-tâches) & \textbf{77.0} & 38.9 & \textbf{56.5}\\
\bottomrule
\end{tabularx}
\caption{Résultats sur le jeu de données SUN RGB-D dataset (images de $224\times224$px). Les métriques utilisées sont le taux de bonne classification, la moyenne du rapport intersection sur union (I/U) et la précision moyenne.}
\label{tab:sun_results}
\end{table}
Nous indiquons dans le~\cref{tab:sun_results} les résultats détaillés de segmentation sur le jeu de données SUN RGB-D. Le passage à un modèle multi-tâche améliore légèrement la précision moyenne et le taux moyen de bonne classification, contre une très faible diminution du rapport I/U. Ces résultats montrent que l'utilisation de la régression des cartes de distances s'étend également à des architectures multi-modales à double entrée. En outre, nos résultats sont comparables à ceux obtenus par~\cite{qi_3d_2017} utilisant un réseau de neurones convolutif sur le graphe 3D de la scène, qui utilise donc une information plus riche.
%We report in~\cref{tab:sun_results} test results on the SUN RGB-D dataset. Switching to the multi-task setting improves the overall accuracy and the average precision by respectively 0.33 and 1.06 points, while very slightly decreasing the average IoU. This shows that the distance transform regression also generalizes to a multi-modal setting on a dual-stream network.

% \paragraph{Data Fusion Contest 2015}

% \begin{table*}[!t]
% \begin{tabularx}{\textwidth}{Y c c c c c c c c c}
% % 86.66544019909854%
% % ---
% % F1Score :
% % roads: 0.8404566588659725
% % buildings: 0.8221298896312921
% % low veg.: 0.8223702385450992
% % trees: 0.6909792712073924
% % cars: 0.7927102685980361
% % clutter: 0.6577661257351193
% % boat: 0.567996691457946
% % water: 0.9892707731278347
% %
% % 87.3143965%
% % ---
% % roads: 0.8404094014740641
% % buildings: 0.8171302465849813
% % low veg.: 0.8387622103316779
% % trees: 0.8004487494080489
% % cars: 0.8027419710959867
% % clutter: 0.6925006989638699
% % boat: 0.5082705077291362
% % water: 0.9894498184605314
% \toprule
% Method & OA & Roads & Buildings & Low veg. & Trees & Cars & Clutter & Boat &  Water\\
% \midrule
% AlexNet (patch-based)~\cite{campos-taberner_processing_2016} & 83.32 & 79.10 & 75.60 & 78.00 & 79.50 & 50.80 & 63.40 & 44.80 & 98.20\\
% SegNet (classification) & 86.67 & \textbf{84.05} & \textbf{82.21} & 82.24 & 69.10 & 79.27 & 65.78 & \textbf{56.80} & 98.93\\
% SegNet (multi-task) & \textbf{87.31} & 84.04 & 81.71 & \textbf{83.88} & \textbf{80.04} & \textbf{80.27} & \textbf{69.25} & 50.83 & \textbf{98.94}\\
% \bottomrule
% \end{tabularx}
% \caption{Results on the Data Fusion Contest 2015 dataset. We report F1 scores per class and the overall accuracy (OA).}
% \label{tab:dfc_results}
% \end{table*}

% \cref{tab:dfc_results} details the results on the Data Fusion Contest 2015 dataset compared to the best result from the original benchmark~\cite{campos-taberner_processing_2016}. Most classes benefit from the distance transform regression, with the exception of the ``boat'' class. The overall accuracy is improved by 0.64\% in the multi-task setting. Similarly to the Potsdam dataset, trees and low vegetation strongly benefit from the distance transform regression. Indeed, vegetation is often annotated as closed shapes even if it is possible to see what lies underneath. Therefore, filter responses to the pixel spectrometry can be deceptive. Learning distances forces the classifier to integrate spatial features into the decision process.

\paragraph{CamVid}

\begin{table*}[!t]
\setlength{\tabcolsep}{1pt}
\begin{tabularx}{\textwidth}{Y c c c c c c c c c c c c c}
\toprule
Méthode & I/U & \% classif. & Bâtiments & Arbres & Ciel & Voiture & Panneau & Route & Piéton & Barrière & Poteau & Trottoir & Cycliste\\
\midrule
SegNet~\cite{badrinarayanan_segnet:_2017} & 46.4 & 62.5 & 68.7 & 52.0 & 87.0 & 58.5 & 13.4 & 86.2 & 25.3 & 17.9 & 16.0 & 60.5 & 24.8\\
DeepLab-LFOV~\cite{l._c._chen_deeplab:_2017} & 61.6 & -- & 81.5 & 74.6 & 89.0 & \textbf{82.2} & 42.3 & 92.2 & 48.4 & 27.2 & 14.3 & 75.4 & 50.1\\
DenseNet56~\cite{jegou_one_2017} & 58.9 & 88.9 & 77.6 & 72.0 & 92.4 & 73.2 & 31.8 & 92.8 & 37.9 & 26.2 & 32.6 & 79.9 & 31.1\\
DenseNet103~\cite{jegou_one_2017} & \textbf{66.9} & \textbf{91.5} & \textbf{83.0} & \textbf{77.3} & \textbf{93.0} & 77.3 & \textbf{43.9} & \textbf{94.5} & \textbf{59.6} & 37.1 & \textbf{37.8} & \textbf{82.2} & 50.5\\
\midrule
PSPNet* (classification) & 60.3 & 89.3 & 74.7 & 64.1 & 89.0 & 71.8 & 36.6 & 90.8 & 44.5 & 38.5 & 25.4 & 77.4 & 50.3\\
PSPNet* (multi-tâches) & 62.2 & 90.0 & 76.2 & 66.4 &  88.8 & 78.0 & 37.6 & 90.7 & 47.2 & \textbf{40.1} & 28.6 & 78.9 & \textbf{51.2}\\
%PSPNet (multi-tâches mask) & 60.0 & 89.8 & 75.6 & 67.1 & 89.6 & 71.4 & 37.3 & 92.8 & 44.4 & 36.1 & 27.6 & 75.7 & 42.6\\
\bottomrule
\end{tabularx}
\caption{Résultats sur le jeu de données CamVid incluant le rapport d'intersection sur union (I/U) global et pour chaque classe, ainsi que le taux de bonne classification.}
%sl% SDT = multi-task? ou regression seulement?
%sl% as-tu des résultats avec PSPNet standard? (= classification ?)
\label{tab:camvid_results}
\end{table*}

\begin{figure*}[t]
\centering
\begin{subfigure}{0.21\textwidth}
	\includegraphics[width=\textwidth]{images_camvid/camvid_rgb_1.jpg}
\end{subfigure}
\begin{subfigure}{0.21\textwidth}
	\includegraphics[width=\textwidth]{images_camvid/camvid_std_1.png}
\end{subfigure}
\begin{subfigure}{0.21\textwidth}
	\includegraphics[width=\textwidth]{images_camvid/camvid_reg_1.png}
\end{subfigure}
\begin{subfigure}{0.21\textwidth}
	\includegraphics[width=\textwidth]{images_camvid/camvid_gt_1.png}
\end{subfigure}

\begin{subfigure}{0.21\textwidth}
	\includegraphics[width=\textwidth]{images_camvid/camvid_rgb_2.jpg}
\end{subfigure}
\begin{subfigure}{0.21\textwidth}
	\includegraphics[width=\textwidth]{images_camvid/camvid_std_2.png}
\end{subfigure}
\begin{subfigure}{0.21\textwidth}
	\includegraphics[width=\textwidth]{images_camvid/camvid_reg_2.png}
\end{subfigure}
\begin{subfigure}{0.21\textwidth}
	\includegraphics[width=\textwidth]{images_camvid/camvid_gt_2.png}
\end{subfigure}

\begin{subfigure}{0.21\textwidth}
	\includegraphics[width=\textwidth]{images_camvid/camvid_rgb_3.jpg}
\end{subfigure}
\begin{subfigure}{0.21\textwidth}
	\includegraphics[width=\textwidth]{images_camvid/camvid_std_3.png}
\end{subfigure}
\begin{subfigure}{0.21\textwidth}
	\includegraphics[width=\textwidth]{images_camvid/camvid_reg_3.png}
\end{subfigure}
\begin{subfigure}{0.21\textwidth}
	\includegraphics[width=\textwidth]{images_camvid/camvid_gt_3.png}
\end{subfigure}
\caption{Exemple de résultats de segmentation sémantique sur le jeu de données CamVid. De gauche à droite\,: image RVB, PSPNet (classification), PSPNet (multi-tâches), annotations.}
\label{fig:camvid_results}
\end{figure*}

Les résultats sur le jeu de données CamVid sont détaillés dans le~\cref{tab:camvid_results} avec notamment une comparaison à la méthode de~\cite{jegou_one_2017}. Plusieurs exemples qualitatifs sont illustrées dans la~\cref{fig:camvid_results}. Le passage de PSPNet au mode de fonctionnement multi-tâches permet d'améliorer le rapport I/U global de presque 2 points et améliore la majorité des classes, à l'exception du ciel et des routes. Ceci est notamment dû à la présence de pixels non annotés, nombreux aux frontières de ces classes, provoquant la génération de cartes de distances inexactes. Dans l'ensemble, nos résultats sont compétitifs avec les méthodes de l'état de l'art~\cite{jegou_one_2017,l._c._chen_deeplab:_2017}.
%The test results on the CamVid dataset are reported in~\cref{tab:camvid_results} that also includes a comparison with other methods from the state-of-the-art, notably~\cite{jegou_one_2017}. Some examples are shown in~\cref{fig:camvid_results} where the distance transform regression once again produces smoother segmentations. The PSPNet baseline is competitive with those other methods and its mean IoU is improved by 0.5 by switching to the multi-task setting including the distance transform regression. Most classes benefit from the distance transform regression, with the exception of the ``road'' and ``sky'' classes. This is due to the void pixels, that are concentrated on those classes and that result in noisy distance labels.

\subsection{Discussion}

%\paragraph{Hyperparameter tuning}

% \begin{figure}[!t]
% 	\captionsetup[subfigure]{position=b}
% 	\begin{subfigure}{0.49\linewidth}
%     \includegraphics[width=\textwidth]{barplot_vaihingen}
%     \caption{Relative improvement for several values of $\lambda$.}
%     \end{subfigure}
%     \hfill
%     \begin{subfigure}{0.49\linewidth}
%     \includegraphics[width=\textwidth]{lambdaplot_vaihingen}
%     \caption{Evolution of the overall accuracy with respect to $\lambda$.}
%     \end{subfigure}
%     \caption{Exploration of several values for $\lambda$ on the ISPRS Vaihingen dataset.}
%     \label{fig:lambda_space}
% \end{figure}

%In order to better understand how the weight of each loss impacts the learning process, we train several models on the ISPRS Vaihingen dataset using different values for $\lambda$. This adjusts the relative influence of the distance transform regression compared to the cross-entropy classification loss. As reported in~\cref{tab:isprs_results}, we compared the regression + classification framework to both individual regression and classification. It is worth noting that SDT regression alone performs worse than the classification. This justifies the need to concatenate the inferred SDT with the last layers features in order to actually improve the performances.
%Classification alone can be interpreted as $\lambda = 0$.
%As can be seen in the results, incorporating the SDT regression by increasing $\lambda$ helps the network significantly. Improvements obtained with several values of $\lambda$ are detailed in~\cref{fig:lambda_space}. There are two visible peaks: one around 0.5 and one around 2. However, these two are not equivalent. The 0.5 peak is unstable and presents a high standard deviation in overall accuracy, while the $\lambda = 2$ peak is even more robust than the traditional classification.
%Interestingly, this value is equivalent to rescaling the gradient from the distance regression so that its norm is approximately equal to the gradient coming from the classification. Indeed, experiments show that there is a ratio of 2 between both gradients and that they decrease roughly at the same speed during training. Therefore, it seems that better results are obtained when both tasks are given similar weights in the optimization process.
%Nonetheless, all values of $\lambda$ in the test range improved the accuracy and reduced its standard deviation, making it a fairly easy hyperparameter to choose.

%Finally, we also investigate the impact of using the distance transform regression compared to performing the regression on the label binary masks, which can be seen as a clipped-SDT with a threshold of 1. We experimented this on CamVid, as reported in~\cref{tab:camvid_results}. Using the L1 regression on the masks does not improve the segmentation and even worsens it on many classes. This is not surprising, as the regularization brought by the SDT regression relies on spatial cues that are absent from the binary masks.

%\paragraph{Effect of the multi-task learning}

L'intégration de la régression des cartes de distances dans un cadre d'optimisation multi-tâches permet d'améliorer et de lisser la structure spatiale des segmentations prédites par le réseau. En particulier, le modèle est contraint d'apprendre la notion de proximité spatiale d'un pixel par rapport à des classes voisines. En particulier, dans le cas des images aériennes, des arbres dont le feuillage tombe en hiver peuvent révéler le sol. La réponse spectrale des filtres correspond alors à un mélange de texture, bien que l'annotation recherchée corresponde à l'enveloppe convexe de l'arbre. La régression des cartes de distances permet d'orienter le réseau vers des recherches de structure géométriques, moins dépendantes de la radiométrie locale. En outre, cela permet de limiter la présence du bruit de classification poivre et sel qui est habituellement corrigé par des modèles graphiques a posteriori.
%The multi-task learning incorporating the distance transform regression in the semantic segmentation model helps the network to learn spatial structures. More precisely, it constrains the network not only to learn if a pixel is in or out a class mask, but also the Euclidean distance of this pixel w.r.t the mask. This information can be critical when the filter responses are ambiguous. For example, trees from birdviewX might reveal the ground underneath during the winter, as there are no leaves, although annotations still consider the tree to have a shape similar to a disk. Spatial proximity helps in taking these cases into account and removing some of the salt-and-pepper classification noise that it induces, as shown on the ISPRS Vaihingen and Potsdam and DFC2015 datasets.
%Moreover, as the network has to assign spatial distances to each pixel w.r.t the different classes, it also learns helpful cues regarding the spatial structure underlying the semantic maps. As illustrated in~\cref{fig:inria_results}, the predictions become more coherent with the original structure, with sharper boundaries and less holes when shapes are supposed to be closed.

\bibliographystyle{plainnat}
\bibliography{Chapitre6/Biblio}
