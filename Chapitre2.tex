\chapter{Cartographie automatisée d'images aériennes}
	\citationChap{}{}
	\minitoc
	\newpage

%%%%%%%%%%%%%%%%%%%%%%%%%%%%%%%%%%%%%%%%%%%%%%%%%%%%%%%%%%%%%%%%%%%%%%%%%%%%%%%%%%%%%%%%%%%%%

\section{Méthodes de cartographie d'images de télédétection}

Il est possible de distinguer deux étapes dans le processus de classification de données. La première étape, dite d'extraction de caractéristiques, consiste à transformer la donnée pour la projeter dans un espace de représentation adapté. La seconde consiste à la classification à proprement parler, c'est-à-dire à la séparation de l'espace ainsi formé en sous-ensemble disjoints.

Par exemple, dans le cas d'une machine à vecteur de support linéaire, la classification s'effectue en déterminant les hyperplans permettant de séparer au mieux les données. L'espace de représentation doit donc être, si possible, linéairement séparable. Il s'agit alors de trouver des caractéristiques (c'est-à-dire une projection) adaptées.

\subsection{Classification pixelliques}

Pixel = unité atomique de surface
Pur, en tout cas la donnée la moins mélangée disponible
Extraire des attributs du pixel (éventuellement en prenant en compte ses voisins)
Puis classification
Inconvénient: beaucoup de pixels avec l'augmentation de la résolution spatiale, donc ça peut devenir très lent
En outre, pas d'approche objet possible

\subsection{Classification par région}

Pré-segmentation pour définir des régions homogènes dans l'image
Soit segmentation "objet" : Felzenswalb, eCognition
Soit segmentation "superpixels": SLIC, Quickshift
On a aussi des segmentations hiérarchiques "HSEG" etc. qui permettent d'extraire des attributs multi-échelles naturellement

Ensuite, on peut extraire des attributs sur ces régions et classer l'image région par région.

\subsection{Modèles statistiques usuels}

Random Forest, SVM, AdaBoost, XGBoost...

\section{Réseaux de neurones profonds}

\subsection{Réseaux de neurones artificiels convolutifs}

Enchaînement de convolutions + activations non linéaires

Partie finale entièrement connectée : classifieur

L'extraction des caractéristiques et la classification se font de manière conjointe : optimisation de bout en bout

\subsection{Réseaux de neurones entièrement convolutifs}

On enlève la partie entièrement connectée pour en faire une partie convolutive

Avantage : prédiction spatiale et non plus unidimensionnelle
Avantage : plus de taille fixée d'image
Avantage : on fait automatiquement la classification pixellique !

\subsection{Application à la cartographie sémantique}

Transposition de SegNet et autres aux images aériennes : pas de verrou particulier

Fenêtre glissante, traitement par batchs, random sampling, augmentation de données

\section{Évaluation des modèles}

Scores F1, précision, rappel, taux de bonne classification, intersection sur union

\section{Bases de données}

\subsection{ISPRS 2D Semantic Labeling}

\subsection{ONERA Christchurch}

\subsection{Data Fusion Contest 2015}

\subsection{Inria Aerial Image Labeling}
	
\bibliographystyle{francaissc}
\bibliography{Chapitre2/Biblio}
