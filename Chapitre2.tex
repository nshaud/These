\chapter{Cartographie automatisée d'images aériennes}
	\citationChap{}{}
	\minitoc
	\newpage

%%%%%%%%%%%%%%%%%%%%%%%%%%%%%%%%%%%%%%%%%%%%%%%%%%%%%%%%%%%%%%%%%%%%%%%%%%%%%%%%%%%%%%%%%%%%%

\section{Méthodes de cartographie d'images de télédétection}

Il est possible de distinguer deux étapes dans le processus de classification de données. La première étape, dite d'extraction de caractéristiques, consiste à transformer la donnée pour la projeter dans un espace de représentation adapté. La seconde consiste à la classification à proprement parler, c'est-à-dire à la séparation de l'espace ainsi formé en sous-ensemble disjoints.

Par exemple, dans le cas d'une machine à vecteur de support linéaire, la classification s'effectue en déterminant les hyperplans permettant de séparer au mieux les données. L'espace de représentation doit donc être, si possible, linéairement séparable. Il s'agit alors de trouver des caractéristiques (c'est-à-dire une projection) adaptées.

\subsection{Classification pixelliques}

\subsection{Classification par région}

\subsection{Modèles statistiques usuels}

\section{Réseaux de neurones profonds}

\subsection{Réseaux de neurones artificiels convolutifs}

\subsection{Réseaux de neurones entièrement convolutifs}

\subsection{Application à la cartographie sémantique}

\section{Évaluation des modèles}

\section{Bases de données}

\subsection{ISPRS 2D Semantic Labeling}

\subsection{ONERA Christchurch}

\subsection{Data Fusion Contest 2015}

\subsection{Inria Aerial Image Labeling}
	
\bibliographystyle{francaissc}
\bibliography{Chapitre2/Biblio}
