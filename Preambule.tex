%% Manuscrit de thèse de Nicolas Audebert
%% https://nicolas.audebert.at
%%
%% Adapté du template LaTeX de :
%% Copyright (C) 2014 Dorian Depriester
%% http://blog.dorian-depriester.fr
%%
%% This file may be distributed and/or modified under the conditions
%% of the LaTeX Project Public License, either version 1.3c of this
%% license or (at your option) any later version. The latest version
%% of this license is in:
%%
%%    http://www.latex-project.org/lppl.txt
%%
%% and version 1.3c or later is part of all distributions of LaTeX
%% version 2006/05/20 or later.
%!TEX root = Manuscrit.tex


%%%%%%%%%%%%%%%%%%%%%%%%%%%%%%%%%%%%%%%%
%           Liste des packages         %
%%%%%%%%%%%%%%%%%%%%%%%%%%%%%%%%%%%%%%%%\usepackage[utf8]{inputenc}		% LaTeX, comprend les accents !

%% Standalone
\usepackage{standalone}

%% Réglage des fontes et typo
\usepackage[T1]{fontenc}
\usepackage[utf8]{inputenc}		% LaTeX, comprend les accents !
%\usepackage{chapterbib}
%\renewcommand{\bibsection}{\section{Références}}	% Met les références biblio dans un \section (au lieu de \section*)

\usepackage[french]{babel}
\usepackage{lmodern}
\usepackage{ae,aecompl}										% Utilisation des fontes vectorielles modernes
%\usepackage[upright]{fourier}
\usepackage[frenchstyle]{kpfonts}
%\usepackage{mathpazo}
\usepackage[defaultsans]{droidsans}    % Fonte sans serif pour la page de garde
%\usepackage[sfdefault]{AlegreyaSans} % Autre fonte sans serif
\usepackage{csquotes}
\usepackage[babel,kerning,protrusion,expansion]{microtype} % Ajuste finement les espaces
% En outre, les options babel et kerning permettent d'utiliser ':' avec babel en français et cleveref dans les \caption (en théorie ':' est un caractère actif avec babel)
\usepackage{bbding}

%%%%%%%%%%%%%%%%%%%%%%%%%%%%%%%%%%%%%%%%%%%%%%%%%%%%%%%%%%%%%%%%%%%%%
% Allure générale du document
%\usepackage{enumerate}
%\usepackage{enumitem}
\usepackage[section]{placeins}	% Place un FloatBarrier à chaque nouvelle section
\usepackage{epigraph}
\usepackage[font={small}]{caption}
\usepackage[francais,nohints]{minitoc}	% Mini table des matières, en français
	\setcounter{minitocdepth}{2}	% Mini-toc détaillées (sections/sous-sections)
%\usepackage[notbib]{tocbibind}		% Ajoute les Tables des Matières/Figures/Tableaux à la table des matières
\usepackage[Lenny]{fncychap}		% Jolis en-têtes de chapitre
\usepackage{multicol}
\usepackage[style=numeric,natbib=true,refsection=chapter,backend=biber,backref=true]{biblatex}

%%%%%%%%%%%%%%%%%%%%%%%%%%%%%%%%%%%%%%%%%%%%%%%%%%%%%%%%%%%%%%%%%%%%%
%% Maths
\usepackage{amsmath}			% Permet de taper des formules mathématiques
\usepackage{amssymb}			% Permet d'utiliser des symboles mathématiques
\usepackage{amsfonts}			% Permet d'utiliser des polices mathématiques
\usepackage{nicefrac}			% Fractions 'inline'
\usepackage{amsthm} 		       % Théorèmes
\newtheorem{theorem}{Théorème}
\newtheorem{definition}{Définition}
\newcommand{\diff}{\mathrm{d}} % Dérivation

%%%%%%%%%%%%%%%%%%%%%%%%%%%%%%%%%%%%%%%%%%%%%%%%%%%%%%%%%%%%%%%%%%%%%
%% Tableaux
\usepackage{multirow}			% Permet de combiner plusiueurs ligns
\usepackage{booktabs}			% Jolis tableaux
%\usepackage{colortbl}
\usepackage{tabularx}			% Tableaux avec largeur de colonne automatique
\newcolumntype{Y}{>{\centering\arraybackslash}X} % Colonne de largeur automatique et centrée
\usepackage{diagbox}
\usepackage{etoolbox}
	\appto\TPTnoteSettings{\footnotesize}
\addto\captionsfrench{\def\tablename{{\textsc{Tableau}}}}	% Renome 'table' en 'tableau'
%% Couleurs
\usepackage[table,dvipsnames]{xcolor}

%%%%%%%%%%%%%%%%%%%%%%%%%%%%%%%%%%%%%%%%%%%%%%%%%%%%%%%%%%%%%%%%%%%%%
%% Eléments graphiques
\usepackage{graphicx}			% Permet l'inclusion d'images
\usepackage{subcaption}
%\usepackage{pdfpages}
%\usepackage{rotating}
\usepackage{pgfplots}
	\usepgfplotslibrary{groupplots}
\usepackage{tikz}
	\usetikzlibrary{backgrounds,automata,3d,calc,patterns}
	\pgfplotsset{width=7cm,compat=1.3}
	\tikzset{every picture/.style={execute at begin picture={
   		\shorthandoff{:;!?};}
	}}
	\pgfplotsset{every linear axis/.append style={
		/pgf/number format/.cd,
		use comma,
		1000 sep={\,},
	}}
\usepackage{eso-pic}
\usepackage{import}

%%%%%%%%%%%%%%%%%%%%%%%%%%%%%%%%%%%%%%%%%%%%%%%%%%%%%%%%%%%%%%%%%%%%%
%% Mise en forme du texte
\usepackage{xspace}
\usepackage{siunitx}
\DeclareSIUnit\px{px}

\usepackage{textcomp}
\usepackage{array}
\usepackage{hyphenat}

%%%%%%%%%%%%%%%%%%%%%%%%%%%%%%%%%%%%%%%%%%%%%%%%%%%%%%%%%%%%%%%%%%%%%%
%% Packages pour l'aide à l'écriture
\usepackage[french]{todonotes}

%%%%%%%%%%%%%%%%%%%%%%%%%%%%%%%%%%%%%%%%%%%%%%%%%%%%%%%%%%%%%%%%%%%%%
%% Navigation dans le document
\usepackage[pdftex,pdfborder={0 0 0}]{hyperref}	% Créera automatiquement les liens internes au PDF
					% Doit être chargé en dernier (Sauf exceptions ci-dessous)

\hypersetup{colorlinks=true,
			linkcolor=[rgb]{0,0,0.5},
			citecolor=[rgb]{0.5,0,0}}

%%%%%%%%%%%%%%%%%%%%%%%%%%%%%%%%%%%%%%%%%%%%%%%%%%%%%%%%%%%%%%%%%%%%%
%% Packages qui doivent être chargés APRES hyperref
\usepackage[top=2.5cm, bottom=2cm, left=3cm, right=2.5cm,
			headheight=15pt]{geometry}

\usepackage{fancyhdr}			% Entête et pieds de page. Doit être placé APRES geometry
	\pagestyle{fancy}		% Indique que le style de la page sera justement fancy
	\lfoot[\thepage]{} 		% gauche du pied de page
	\cfoot{} 			% milieu du pied de page
	\rfoot[]{\thepage} 		% droite du pied de page
	\fancyhead[RO, LE] {}

\usepackage[acronym,toc,numberedsection,ucmark]{glossaries}
	\newglossary[nlg]{notation}{not}{ntn}{Notation} % Création d'un type de glossaire 'notation'
	\makeglossaries
	\loadglsentries{Glossaire}			% Utilisation d'un fichier externe pour la définition des entrées (Glossaire.tex)
\makeglossaries
\usepackage[capitalize,french]{cleveref}

\addbibresource{Chapitre1/Historique.bib}
\addbibresource{Chapitre2/Biblio.bib}
\addbibresource{Chapitre3/Biblio.bib}
\addbibresource{Chapitre4/Biblio.bib}
\addbibresource{Chapitre5/Biblio.bib}
\addbibresource{Chapitre6/Biblio.bib}
\addbibresource{Annexes/Datasets.bib}
